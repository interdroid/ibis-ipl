\documentclass[10pt]{article}

\usepackage{graphicx}
\usepackage{url}

\newcommand{\mysection}[1]{\section{#1}\label{#1}}
\newcommand{\mysubsection}[1]{\subsection{#1}\label{#1}}
\newcommand{\mysubsubsection}[1]{\subsubsection{#1}\label{#1}}
\newcommand{\remark}[1]{[\emph{#1}]}

\begin{document}

\title{Ibis Users Guide}

\author{The Ibis Group}

\maketitle

\section{Introduction}

This manual describes the steps required to run an application that uses
 the Ibis communication library. How to create such an application
is described in the IPL Programmers manual.

A central concept in Ibis is the \emph{Pool}. A Pool consists of one or
more Ibis instances, usually running on different machines. Each pool is
generally made up of Ibisses running a single distributed applications.
Ibisses in a pool can communicate with each other, and, using the
registry mechanism present in Ibis, can search for other Ibisses in the
same pool, get notified of Ibisses joining the pool, etc. To
coordinate Ibis pools a so-called \emph{Ibis Server} is used.

\section{The Ibis Server}

The Ibis Server is the Swiss-army-knife server of the Ibis project.
Services can be dynamically added to the server. By default, the Ibis
communication library comes with several services. The registry service
keeps track of pools, and can track multiple pools at the same time.
The server also allows Ibisses to route traffic over the server if no
direct connection is possible between two instances due to firewalls or
NAT boxes.

The Ibis server is started with the \texttt{ibis-server} script which is
located in the Ibis \texttt{bin} directory.  Before starting an Ibis
application, an Ibis server needs to be running on a machine that is
accessible from all nodes participating in the Ibis run. The server
listens to a TCP port. The port number can be specified using the
\texttt{--port} command line option to the \texttt{ibis-server} script.

The Ibis Server needs to be reachable from every Ibis
instance. Since sometimes this is not possible due to firewalls,
additional \emph{hubs} can be started, creating a routing infrastructure
for the Ibis instances. These hubs can be started by using ibis-server
script with the \texttt{--hub-only} option. See the \texttt{--help}
option of the ibis-server script for more information. The Ibis
instances then need to be given the address of a reachable hub, as well
as the address of the Ibis server.

TODO:Explain about hubs and the server

TODO:Explain semantics and syntax of properties 

TODO:Explain script is simply an example

\mysubsection{Running an Ibis Application}

To illustrate running an Ibis application we will use a simple Hello
world application. This application is started on a single machine.

An Ibis instance is started with the \texttt{ibis-run} script which
is located in the Ibis \texttt{bin} directory.  This \texttt{ibis-run}
script is called as follows:
\begin{center}
\texttt{ibis-run} \emph{java-flags class params}
\end{center}

The \texttt{ibis-run} script adds the jar files
from the Ibis lib directory to your classpath and then starts Java. The
scripts needs the IBIS\_HOME environment variable to be set to the
location of the Ibis distribution.

\mysubsection{Running the example}
To run the application an ibis-server is needed. Start a shell and
run the \texttt{ibis-server} script:
\noindent
{\small
\begin{verbatim}
$ $IBIS_HOME/bin/ibis-server --events
\end{verbatim}
}
\noindent

The \texttt{--events} option will make the server print events (such as
Ibisses joining and leaving the pool). Run the script with the
\texttt{--help} option for a complete list of options.

Next, we will start the application twice. One instance will act as a  
"server", and one a "client". The application will determine who is
the server and who is the client automatically. Therefore we can simply start the application using
the same command line. In two different shells type:

\noindent
{\small
\begin{verbatim}
$ $IBIS_HOME/bin/ibis-run \
    -Dibis.server.address=localhost -Dibis.pool.name=foo \
    ibisApps.hello.Hello
\end{verbatim}
}
\noindent
Now, you should have the two running instances of your application. One of them should print:

\noindent {\small \begin{verbatim} Server received: Hi there
\end{verbatim} } \noindent 

The \texttt{ibis.pool.name} value can be any random string.  It
identifies your run. The \texttt{ibis.server.address} property should be
set to the machine the Ibis server is running on. In this case, we use
\texttt{localhost}.

If you don't use the \texttt{ibis-run} script, you have to set the classpath
and some additional properties. In both shells, type:

\noindent
{\small
\begin{verbatim}
$ java \
    -cp $IBIS_HOME/lib/ipl.jar:$IBIS_HOME/lib/ipl-app.jar \
    -Dibis.impl.path=$IBIS_HOME/lib \
    -Dibis.server.address=localhost \
    -Dibis.pool.name=bla \
    -Dlog4j.configuration=file:$IBIS_HOME/log4j.properties \
    ibisApps.hello.Hello
\end{verbatim}
}
\noindent
The classpath specifies the jar-files that are explicitly used:
\texttt{ipl-app.jar} contains the example code, and \texttt{ipl.jar} contains
the Ibis IPL. All other jar-files are loaded by the Ibis classloader, which
searches the directory specified by the \texttt{ibis.impl.path} property
(and its sub-directories) for any needed jar-file.
Additionally, a log4j configuration file is specified, which determines the
number of status messages Ibis prints. The file provided in the Ibis
distribution will make Ibis print
warnings and errors to standard output.

TODO:Explain about ibis.impl.path classloader some more

\mysection{Properties}

Ibis can be configured using several Java system properties. The
\texttt{ibis.properties.example} file in the Ibis distribution has a complete
list, but the most commonly used properties are:

\begin{description}

\item[ibis.pool.name] String: name of the pool this Ibis belongs to
\item[ibis.server.address] Address of the central Ibis server
\item[ibis.impl.path] Path used to find Ibis implementations
\item[ibis.hub.addresses] Comma seperated list of hub addresses. The
server address is appended to this list, and thus is the default hub if
no hub is specified.
\item[ibis.pool.size] Integer: size of the pool this Ibis belongs to.
Only used in a so called closed world setting.

\end{description}

\mysection{Further Reading}

The Ibis web page \url{http://www.cs.vu.nl/ibis} lists all
the documentation and software available for Ibis, including papers, and
slides of presentations.

For detailed information on developing an Ibis application see the
Programmers Manual, available in the docs directory of the Ibis
distribution

\end{document}
