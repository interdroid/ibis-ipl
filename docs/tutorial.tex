After giving a high level overview of the functionality of Ibis, we will
now give some examples of applications which use the IPL. All the
examples used here can also be found in the \texttt{examples} directory
of the distribution.

\subsection{Hello}

\begin{figure}[t]
\lstset{language=Java,basicstyle=\footnotesize,breaklines=false}
\begin{lstlisting}[numbers=left, numbersep=3pt]
package ibis.ipl.examples;

import ibis.ipl.*;

public class Hello {
    PortType portType = new PortType(PortType.COMMUNICATION_RELIABLE,
            PortType.SERIALIZATION_DATA, PortType.RECEIVE_EXPLICIT,
            PortType.CONNECTION_ONE_TO_ONE);

    IbisCapabilities ibisCapabilities = new IbisCapabilities(
            IbisCapabilities.ELECTIONS_STRICT);

    private void server(Ibis myIbis) throws IOException {

        // Create a receive port and enable connections.
        ReceivePort receiver = myIbis.createReceivePort(portType, "server");
        receiver.enableConnections();

        // Read the message.
        ReadMessage r = receiver.receive();
        String s = r.readString();
        r.finish();
        System.out.println("Server received: " + s);

        // Close receive port.
        receiver.close();
    }

    private void client(Ibis myIbis, IbisIdentifier server) throws IOException {

        // Create a send port for sending requests and connect.
        SendPort sender = myIbis.createSendPort(portType);
        sender.connect(server, "server");

        // Send the message.
        WriteMessage w = sender.newMessage();
        w.writeString("Hi there");
        w.finish();

        // Close ports.
        sender.close();
    }

    private void run() throws Exception {
        // Create an ibis instance.
        Ibis ibis = IbisFactory.createIbis(ibisCapabilities, null, portType);

        // Elect a server
        IbisIdentifier server = ibis.registry().elect("Server");

        // If I am the server, run server, else run client.
        if (server.equals(ibis.identifier())) {
            server(ibis);
        } else {
            client(ibis, server);
        }

        // End ibis.
        ibis.end();
    }

    public static void main(String args[]) {
        try {
            new Hello().run();
        } catch (Exception e) {
            e.printStackTrace(System.err);
        }
    }
}
\end{lstlisting}
\caption{Complete source of the Hello program}
\label{hello_source}
\end{figure}

The first example is a very simple Hello World type application. See
Figure~\ref{hello_source} for the complete source. As you can see it is
under a hundred lines of code. This application is meant to be started
twice. One instance will act as a server and one as a client. The client
sends a message to the server. Which instance is the server and which is
the client is done using an election.

The application is split up into several parts. First, the capabilities
needed from Ibis are defined in two global variables. At the very bottom
of the file is the \texttt{main} function. Main creates a object of this
type, and calls \texttt{run} on this object. Run initialized Ibis, and
determines if this is the client or the server. It then calls either the
\texttt{server} or the \texttt{client} method.

We will now explain this application line-by-line. The example starts
with declaring the package on line 1. It then imports all classes from
the IPL on line 3, and start the \texttt{Hello} class on line 5. Since
this application will send messages, it will need to create ports, and
thus a port type. The declaration on line 6-8 creates a port type
suitable for this application. We select
\texttt{COMMUNICATION\_RELIABLE} so we are sure the message will arrive,
\texttt{SERIALIZATION\_DATA} so we are able to send primitive types, not
only bytes (we want to send a String). \texttt{RECEIVE\_EXPLICIT}
denotes we are going to explicitly call the \texttt{receive()} method of
the receive port, and finally \texttt{CONNECTION\_ONE\_TO\_ONE} selects
the simplest communication pattern, where a single sendport is connected
to a single receive port.

Next, we also need a list of all the capabilities we need from our Ibis
itself. Since we would like to use reliable elections, we include
\texttt{ELECTIONS\_STRICT}.

\subsection{Upcalls}

\subsection{Registry}

\subsection{One-to-Many}

\subsection{Many-to-One}

\subsection{Client-Server}

\subsection{Client-Server with Objects}

