After giving a high level overview of the functionality of Ibis, we will
now give some examples of applications which use the IPL. All the
examples used here can also be found in the \texttt{examples} directory
of the distribution. 

\subsection{Hello}
\begin{figure}[p]
\begin{lstlisting}[language=Java,numbers=left, numbersep=3pt]
package Ibis.ipl.examples;

import ibis.ipl.*;

public class Hello {
    PortType portType = new PortType(PortType.COMMUNICATION_RELIABLE,
            PortType.SERIALIZATION_DATA, PortType.RECEIVE_EXPLICIT,
            PortType.CONNECTION_ONE_TO_ONE);

    IbisCapabilities ibisCapabilities = new IbisCapabilities(
            IbisCapabilities.ELECTIONS_STRICT);

    private void server(Ibis myIbis) throws IOException {

        // Create a receive port and enable connections.
        ReceivePort receiver = myIbis.createReceivePort(portType, "server");
        receiver.enableConnections();

        // Read the message.
        ReadMessage r = receiver.receive();
        String s = r.readString();
        r.finish();
        System.out.println("Server received: " + s);

        // Close receive port.
        receiver.close();
    }

    private void client(Ibis myIbis, IbisIdentifier server) throws IOException {

        // Create a send port for sending requests and connect.
        SendPort sender = myIbis.createSendPort(portType);
        sender.connect(server, "server");

        // Send the message.
        WriteMessage w = sender.newMessage();
        w.writeString("Hi there");
        w.finish();

        // Close ports.
        sender.close();
    }

    private void run() throws Exception {
        // Create an ibis instance.
        Ibis ibis = IbisFactory.createIbis(ibisCapabilities, null, portType);

        // Elect a server
        IbisIdentifier server = ibis.registry().elect("Server");

        // If I am the server, run server, else run client.
        if (server.equals(ibis.identifier())) {
            server(ibis);
        } else {
            client(ibis, server);
        }

        // End ibis.
        ibis.end();
    }

    public static void main(String args[]) {
        try {
            new Hello().run();
        } catch (Exception e) {
            e.printStackTrace(System.err);
        }
    }
}
\end{lstlisting}
\caption{Complete source of the Hello program}
\label{hello_source}
\end{figure}

The first example is a very simple Hello World type application. This is
also the example used in the user's guide. So, try to get it to run to
see what it is supposed to do. See Figure~\ref{hello_source} for the
complete source of the Hello application. It is under a hundred lines of
code. This application is meant to be started twice. One instance will
act as a server and one as a client. The client sends a message to the
server. Which instance is the server and which is the client is
determined using an election.

The application is split up into several parts. First, the capabilities
needed from Ibis are defined in two global variables. At the very bottom
of the file is the \texttt{main} method. Main creates a object of this
type, and calls \texttt{run} on this object. Run initializes Ibis, and
determines if this is the client or the server. It then calls either the
\texttt{server} or the \texttt{client} method.

We will now explain this application line-by-line. The example starts
with declaring the package on line 1. It then imports all classes from
the IPL on line 3, and declares the \texttt{Hello} class on line 5. Since
this application will send messages, it will need to create ports, and
thus a port type. The declaration on lines 6-8 creates a port type
suitable for this application. 

The constructor of the \texttt{PortType} class requires us to specify
which capabilities we want to set. We select
\texttt{COMMUNICATION\_RELIABLE} so we are sure the message will arrive
and \texttt{SERIALIZATION\_DATA} so we are able to send primitive types,
not only bytes, since we want to send a string. \texttt{RECEIVE\_EXPLICIT}
denotes we are going to explicitly call the \texttt{receive()} method of
the receive port, and finally \texttt{CONNECTION\_ONE\_TO\_ONE} selects
the simplest communication pattern, where a single send port is connected
to a single receive port.
Next, we also need a list of all the capabilities we need from our Ibis
itself. Since we would like to use reliable elections, we include
\texttt{ELECTIONS\_STRICT}.

The main entry point of this class is the \texttt{main} method, defined
on lines 62-67. It simply creates a object of the Hello type and calls
the \texttt{run} method, defined on lines 44-60. The Ibis instance used
by the applications is created using the \texttt{IbisFactory} class on
line 46. Ibises cannot be created directly, but must be made using a
call to this factory. The \texttt{createIbis} method takes several
parameters. First is the \texttt{capabilities} object of this Ibis.
Next, a \texttt{RegistryEventHandler} must be passed if needed. This
example does not, see the Registry example below for this functionality.
Lastly, the \texttt{createIbis} method requires a list of all the port
types needed by the application. We only have one, which we defined
previously.

When the Ibis instance is created, it is automatically added to the pool
specified with the properties set at the command line (see the
user's guide). So, we can now start using the registry, make
connections, etc. This application starts by determining which instance
will be the server. This is done via an election on line 49. Each
election in Ibis has a name, denoted by a string. So, as each instance
of this application calls the \texttt{elect} method of the registry with
the same parameter ("Server" in this case), one of these Ibises will
"win" this election. The election returns the winner. On line 52 we
compare the winner with our own identifier. If we won the election, the
\texttt{server} method is called on line 53, if we didn't win we are the
client and we call the \texttt{client} method on line 55.

The \texttt{server} method, starting on line 13, waits for a message
from the client and prints it. So, to receive a message we first have to
create a receive port(line 16). Both send and receive ports are created
by calling one of the \texttt{create} methods of the Ibis instance. In
this case, we use the \texttt{createReceivePort} method which has two
parameters: one for the port type, and one for the name of this
receive port. This name must be unique for this Ibis instance, and can
be used to connect to this port. We name our port "server".

After a receive port has been created, it does not automatically start
handling new connections. This allows an application to initialize
itself properly before any connections come in from other Ibises. Since
we don't have anything else to initialize we simple enable incoming
connections immediately using the \texttt{enableConnections} method on
line 17.

Next, we will receive the message. We specified we were going to use
explicit receive in the capabilities of our port type, so we call the
\texttt{receive} method on our receive port(line 20) to wait for a message. Once
a message has been received we read the data from the message (we know
it is a string) by calling the \texttt{readString} method on the read
message. Afterwards, we signal Ibis we are now done with this message by
calling \texttt{finish} on the message. Finally, we print the message on
line 23. Since we are only receiving a single message our server is now
done, and closes the receive port on line 26.

On the client side, we will have to connect to the server, and send it a
message containing a string. So, we first create a send port on line 32
by calling one of the create methods of our Ibis instance. The method we
call expects a port type (this must be equal to the port type of the receive
port we want to connect to). After creating the port we connect our send
port to the receive port. We use the identifier of the server we acquired
at the election, and the name of the port at the server which we know to
be "server".

After the port has been connected, we can now send a message. We ask the
send port for a new write message on line 36, and write a string to it
on line 37. Since we are only writing this one string we finalize the
message on line 38. This completes the work of the client. So, we close
our send port on line 41.

If everything worked as planned, the server should now print the message
it received from the client. Both the server and the client exit their
respective methods and return to the \texttt{run} method on line 57. On
line 58 they both end their Ibis instance. This will make these two
instances leave the pool, ending with a empty pool.

\subsection{Upcalls}

\begin{figure}[ht]
\lstset{basicstyle=\footnotesize,breaklines=false}
\begin{lstlisting}[language=Java,numbers=left, numbersep=3pt]
public class HelloUpcall implements MessageUpcall {

    PortType portType =
        new PortType(PortType.COMMUNICATION_RELIABLE,
                PortType.SERIALIZATION_DATA, PortType.RECEIVE_AUTO_UPCALLS,
                PortType.CONNECTION_ONE_TO_ONE);

    ...

    /**
     * Function called by Ibis to give us a newly arrived message
     */
    public void upcall(ReadMessage message) throws IOException {
        String s = message.readString();
        System.out.println("Received string: " + s);
    }

    private void server(Ibis myIbis) throws IOException {
        // Create a receive port, pass ourselves as the message upcall
        // handler
        ReceivePort receiver =
            myIbis.createReceivePort(portType, "server", this);
        // enable connections
        receiver.enableConnections();
        // enable upcalls
        receiver.enableMessageUpcalls();
    }
...
\end{lstlisting}
\caption{Excerpt from application that uses upcalls}
\label{hello_upcall_source}
\end{figure}

The next example we will look at is basically the same application as
the first, except this application uses upcalls instead of explicit
receive. Instead of listing the entire application, we only list the
parts that need to be changed to use upcalls in Ibis in
Figure~\ref{hello_upcall_source}. For the complete source, see the
\texttt{examples} directory of the distribution.

On line 1, the first change is visible. The application extends the
\texttt{MessageUpcall} interface from the IPL. This class contains a
single method (\texttt{upcall}) which is called whenever a message is
received by Ibis. If upcalls are used it is not necessary to call the
\texttt{receive} method on a receive port. Instead, Ibis will wait for a
message continually and pass any new messages received to the upcall
handler provided by the user, one by one.

Next, the port type needs to be changed slightly. We no longer include
the RECEIVE\_EXPLICIT capability. Instead we use the
RECEIVE\_AUTO\_UPCALLS capability. The "auto" in this capability means
Ibis can receive messages automatically, without any user intervention.
The RECEIVE\_POLL\_UPCALLS capability would require us to signal the
ibis implementation that a new message might be available.

Lines 13-16 show the upcall method as it is implemented by this
application. Again, we read a string from the message and print it.
Notice there is no call to \texttt{finish} on this message. Whenever the
upcall ends the message is automatically finished. It is possible to
call \texttt{finish} though, signaling Ibis that the next message can now be
received. This might be a good idea if a lot of calculations are done in
the upcall method. If \texttt{finish} is not called, Ibis will not
receive the next message until this upcall exits. Also, when you want
to make calls into Ibis, like creating a new send message, setting up new
connections, etc, you \emph{must} finish the read message first, to
prevent deadlocks in Ibis.

Finally, lines 18 to 27 show the server method of this application. One
difference with the previous application can be seen on line 21, when
the receive port is created. Now, we pass the object which will be
called whenever a message is received  to Ibis. In this case, this is
simply ourselves (the third parameter with a \texttt{this} value). Also,
we now need to no only enable connections to this message in line 24,
but must also enable upcalls in line 26.

\subsection{Registry}

\begin{figure}[p]
\lstset{basicstyle=\footnotesize,breaklines=false}
\begin{lstlisting}[language=Java,numbers=left, numbersep=3pt]
package ibis.ipl.examples;

import ibis.ipl.*;

public class RegistryUpcalls implements RegistryEventHandler {

    IbisCapabilities ibisCapabilities =
        new IbisCapabilities(IbisCapabilities.MEMBERSHIP_TOTALLY_ORDERED);
    
    // Methods of the registry event handler. We only implement the 
    // join/leave/died methods, as signals and elections are disabled
    
    public void joined(IbisIdentifier joinedIbis) {
        System.err.println("Got event from registry: " + joinedIbis
                + " joined pool");
    }

    public void died(IbisIdentifier corpse) {
        System.err.println("Got event from registry: " + corpse + " died!");
    }

    public void left(IbisIdentifier leftIbis) {
        System.err.println("Got event from registry: " + leftIbis + " left");
    }

    public void gotSignal(String signal) {
        // NOTHING
    }

    public void electionResult(String electionName, IbisIdentifier winner) {
        // NOTHING
    }
    
    private void run() throws Exception {
        // Create an ibis instance, pass ourselves as the event handler
        Ibis ibis = IbisFactory.createIbis(ibisCapabilities, this);
        ibis.registry().enableEvents();

        // sleep for 30 seconds
        Thread.sleep(30000);

        // End ibis.
        ibis.end();
    }

    public static void main(String args[]) {
        try {
            new RegistryUpcalls().run();
        } catch (Exception e) {
            e.printStackTrace(System.err);
        }
    }
}
\end{lstlisting}
\caption{Complete source of registry example application}
\label{registry_source}
\end{figure}

After examples of how to send and receive messages, we will now give an
example of an application which uses the registry. This application,
for which the source is listed in Figure~\ref{registry_source}, prints
out whenever an Ibis joins the pool, an Ibis leaves the pool, or an Ibis
dies.

Since this application does not actually send any messages, no port type
is defined. Instead, only the \texttt{MEMBERSHIP\_TOTALLY\_ORDERED}
capability is specified (lines 7-8). This enables the membership
management features of Ibis.

The \texttt{main} method of this application creates an object for this
application, and calls the \texttt{run} method defined on lines 34-44.
This method creates an Ibis on line 36. In this call, two things are
passed. First, the capabilities of this Ibis, in this case only the
single capability. Second, a \texttt{RegistryEventHandler} needs to be
passed to automatically receive registry events such as joins and leaves
of Ibis instances. We pass ourselves as the event handler. The class
implements the \texttt{RegistryEventHandler} interface defined in the
IPL. This interface consists of 5 methods. These methods are implemented
on lines 13 to 32 in this example. Wherever a new Ibis joins the pool,
the \texttt{joined} method is called by the registry. Likewise, when a
Ibis leaves the pool, or when it crashes the \texttt{left} or
\texttt{died} methods are called. In this example, all these events are
handled by simply printing out the event that occurred.

The last two methods, \texttt{gotSignal} and \texttt{electionResult} are
not implemented here, since we have not enabled signals or elections.
However, if we added signals and election to our capabilities, these
methods would be called whenever a signal was received or a new election
result was available.

This application uses upcalls to receive the events of the registry.
However, it is also possible to poll the registry. By enabling a
membership capability, but not passing an event handler, a user can call
the \texttt{joinedIbisis}, \texttt{leftIbisis} and \texttt{diedIbisis}
methods in the \texttt{Registry} class, which return any changes to the
pool since the methods were last called. For an example of such an
application, see the \texttt{RegistryDowncalls} example application.

\subsection{Other Examples}

The Ibis distribution also contains some other examples. These include
a \texttt{OneToMany} example showing a simple broadcast application, and a
\texttt{ManyToOne} example which resembles the HelloUpcall application,
but with multiple client.

The last, application in the \texttt{example} directory is
the \texttt{ClientServer} application. This application uses object
serialization and multiple port types to implement a time server demo.
It also shows the usage of an \emph{ReceivePortIdentifier}. Instead of
passing the Ibis and name of a receive port when connection, a user can
also pass the unique identifier of a receiver port directly.
