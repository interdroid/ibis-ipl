\documentclass[10pt]{article}

\usepackage{html}
\usepackage{graphicx}

\newcommand{\mysection}[1]{\section{#1}\label{#1}}
\newcommand{\mysubsection}[1]{\subsection{#1}\label{#1}}
\newcommand{\mysubsubsection}[1]{\subsubsection{#1}\label{#1}}
\newcommand{\remark}[1]{[\emph{#1}]}

\begin{document}

\title{
{\html {\htmladdnormallink{Ibis} {http://www.cs.vu.nl/ibis/}}}
{\latex {Ibis}}
Portability Layer (IPL) Programmer's Manual}

\author{The Ibis Group}

\maketitle

\section{Introduction}

Ibis is an efficient and flexible Java-based programming environment for Grid
computing, in particular for distributed supercomputing applications.
This manual describes the Ibis Portability Layer (IPL). It also describes
how to run an Ibis application.
\html{
It is also available in
\htmladdnormallink{postscript}{http://www.cs.vu.nl/ibis/progman.ps.gz}.}
\latex{It is available on-line at http://www.cs.vu.nl/ibis/progman.}
Ibis is described in several publications (see Section \ref{Further Reading}).
Rather than giving a detailed overview of what each class and method does,
the aim of this document is to describe how to actually use these classes
and methods.
The \emph{javadoc} subdirectory of the Ibis installation provides
documentation for each class and method (point your favorite browser
to javadoc/index.html file of the Ibis installation).
The Ibis API is also available
\html{\htmladdnormallink{on-line}{http://www.cs.vu.nl/ibis/javadoc}.}
\latex{on-line at http://www.cs.vu.nl/ibis/javadoc}.
In this manual, fragments of an actual Ibis application will be used for
illustration purposes.
Section \ref{An Ibis Application} will discuss a typical Ibis application,
with subsections on each phase of the program.
Section \ref{Compiling and Running an Ibis Application} will discuss how to
actually compile and run this program.

\mysection{Some Ibis concepts}

\mysubsection{The Ibis Portability Layer}

The Ibis Portability Layer (IPL) consists of a set of Java interfaces and
classes that define how an Ibis application can make use of the Ibis components.
The Ibis application does not need to know which specific Ibis implementations
are available.
It just specifies some capabilities that it requires, and the Ibis system
selects the best available Ibis implementation that meets these requirements.
 
\mysubsection{An Ibis Instance}

A loaded IPL implementation is called an \emph{Ibis instantiation}, or 
\emph{Ibis instance}.
An Ibis instance is identified by a so-called
\emph{Ibis identifier}.
An application can find out which Ibis instances are present in the run
by supplying a so-called \emph{RegistryEventHandler}.
This RegistryEventHandler is an object with, among others, a \texttt{joined()}
method which gets called by the Ibis system when a new Ibis instance
joins the run.  The Ibis identifier of this new Ibis is a parameter
to the \texttt{joined()} method.

\mysubsection{Send Ports and Receive Ports}

The IPL provides primitives to set up connections between send and receive
ports.
In general, a sendport can have connections to multiple receive ports,
and a receive port can have multiple incoming connections.
However, the user may specify that a sendport will have only one
outgoing connection, or that a receive port will only have one
incoming connection, allowing Ibis to choose a more efficient
implementation.  A connection is always \emph{unidirectional}, from
a send port to a receive port.

All send and receive ports have a \emph{type} which is represented by an
instance of \texttt{ibis.ipl.PortType}.
To create a connection, an Ibis application must create a send port,
and the instance at the receiving side must create a receive port.
The send port can then be connected to the receive port, provided
that the ports are of the same port type.

To send a message, the Ibis application requests a new write message from
a send port. The write message object has methods to put data in this write
message. Finally, the message is finished by invoking its \texttt{finish()}
method.

To receive a message, the IPL provides two mechanisms:
\begin{description}
\item[explicit receipt]
when a receive port is configured for explicit receipt, a message can be
received with the receive port's blocking \emph{receive} method,
or with its non-blocking \emph{poll} method.
These methods return a \emph{read message} object, from which data can
be extracted using its read methods. The \emph{poll} method may also
return \emph{null}, in case no message is available.
\item[upcalls]
when a receive port is configured for upcalls, the Ibis application provides
an \emph{upcall} method, which is to be called when a message arrives.
The upcall provides the message received as a parameter.
The message contents will be lost when the upcall returns, so the data
in the message must be read in the upcall.
Upcalls are either automatic, or must be polled for explicitly, see Section
\ref{Creating an Ibis Instance}.
\end{description}
\noindent

\mysubsection{Port Types}

Send and receive ports are \emph{typed} by means of a \emph{port type}.
A port type is defined by capabilities.
Port type capabilities that can be configured are, for instance, the
serialization method used, reliability, whether a send port can connect
to more than one receive port, whether more than one send port can connect
to a single receive port, et cetera.
Only ports of the same type can be connected.

\mysubsection{Serialization}

Serialization is a mechanism for converting Java objects
into portable data that can be stored or transferred.
Java has input
(\texttt{java.io.ObjectInputStream})
and output
(\texttt{java.io.ObjectOutputStream})
streams for reading and writing
objects.
Ibis has \texttt{writeObject} and \texttt{readObject} methods in
its messages to accomplish the same effect. 

Sometimes, object serialization is not needed. For that case,
two simpler serialization mechanisms are available: \emph{data
serialization} which allows for sending/receiving data of basic types
and arrays of basic types (similar to \texttt{java.io.DataInputStream}
and \texttt{java.io.DataOutputStream}), and \emph{byte serialization}
which only allows sending/receiving bytes and arrays of bytes.

\mysection{An Ibis Application}

An Ibis application consists of several parts:
\begin{itemize}
\item
Creating an Ibis instance in each instance of the application.
An Ibis application can run on multiple hosts.
On each of these hosts, an Ibis instance must be created.
\item
Setting up communication. Communication setup in Ibis
consists of creating one or more \emph{send ports}, through which messages
can be sent, and creating one or more \emph{receive ports},
through which messages can be received, and creating connections between them.
\item
Actually communicating. A send port is used to create a 
\emph{write message}, which is sent to the receive ports that this send port
is connected to.
\item
Finishing up. Connections must be closed, and each Ibis instance must
be ended.
\end{itemize}
\noindent
The next few subsections will discuss each of these steps in turn,
illustrating them with parts of an RPC-style Ibis application.
This application will have a server and one or more clients. This is a toy
application: the server will receive strings, compute the length of each string,
and send this length back to the client.
The clients will send strings, and receive the length of these strings.
The server will have to do some other work as well, just to make
things a little more interesting.

\subsection{Program Preamble}

Ibis applications need to import classes from the IPL (Ibis
Portability Layer) package, which lives in
\texttt{ibis.ipl}.
We recommend that you simply import all \texttt{ibis.ipl} classes with
one import line:

{\small
\begin{quote}
\begin{verbatim}
import ibis.ipl.*;
\end{verbatim}
\end{quote}
}

\noindent
Of course it is also possible to import only the needed classes, but
this tends to result in a list of 10 or more \texttt{import}s.

Before creating an Ibis instance, we must determine what we want from it.
This is determined by two objects: the required Ibis capabilities, and
the list of port types that are going to be used.

\mysubsubsection{Ibis Capabilities}

Below is a snippet from the RPC example, constructing the required capabilities:
{\small
\begin{quote}
\begin{verbatim}
IbisCapabilities capabilities = new IbisCapabilities(
    IbisCapabilities.ELECTIONS_STRICT,
    IbisCapabilities.WORLDMODEL_CLOSED);
\end{verbatim}
\end{quote}
}

This states that the selected Ibis must support reliable elections
(the toy example will use this to determine who will be the server and
who will be the client). The selected Ibis
must also support the ``closed'' worldmodel, which means
that all participating Ibises join at the start of the run.

\mysubsection{Creating an Ibis Instance}

All instances of a program that want to participate in an Ibis run
must create an Ibis instance.
To create an Ibis instance, the static \texttt{createIbis()} method of the
\texttt{ibis.ipl.IbisFactory} class must be used.
The specification of this method is:
{\small
\begin{quote}
\begin{verbatim}
static Ibis createIbis(IbisCapabilities requiredCapabilities,
		       java.util.Properties properties,
		       boolean addDefaults,
                       RegistryEventHandler handler,
		       PortType... portTypes)
        throws IbisCreationFailedException;
\end{verbatim}
\end{quote}
}

There may be several Ibis implementations available, and
the factory selects the best one for you, based on the
specified required capabilities and port types.
These required capabilities and port types in fact determine what
the user wants the chosen Ibis to be able to do.
The Ibis factory will select an Ibis implementation that can at least
do what is required, and if no such implementation is found, throw an exception.
The possible Ibis capabilities are specified in
\texttt{ibis.ipl.IbisCapabilities} class, as discussed in Section
\ref{Ibis Capabilities}.

The \texttt{properties} can be used to pass some implementation-specific
values on to the implementation.  Optionally, some hard-wired defaults
can be added to the properties.

A registry event handler may be specified,
with a.o. \texttt{joined()} and \texttt{left()} upcalls that get
called when an Ibis instance joins or leaves the run.
In our RPC example, we will not use this, so we
specify \texttt{null} instead.
However, when such a \texttt{RegistryEventHandler}
is used, its \texttt{joined()} upcall is called for every Ibis that joins the
run, including the Ibis being created itself.
Upcalls to the \texttt{RegistryEventHandler} must be explicitly enabled by
invoking the \texttt{enableRegistryEvents()} method of the Ibis
just created. This ensures that the \texttt{RegistryEventHandler} has been
able to do the necessary initializations.

The \texttt{enableResizeUpcalls()} method blocks until the
\texttt{joined()} upcall for this Ibis has been invoked.  Knowing which Ibises
have joined the run, and how many there are, is often useful in dividing
the work. See also section \ref{Which Instance Does What?}.

The last parameters specify the port types that are going to be used
in the application. We will discuss these in more detail in Section
\ref{Ibis Port Types}. For now, we will assume the existence

Now back to our example. In Secion \ref{Ibis Capabilities} we will
discuss the creation of an \texttt{IbisCapabilities} object describing the
required capabilities. For the moment we will just assume it
exists, and create an Ibis instance as follows:
{\small
\begin{quote}
\begin{verbatim}
Ibis ibis = null;
try {
    ibis = Ibis.createIbis(capabilities, null, null, null);
} catch(NoMatchingIbisException e) {
    System.err.println("Could not find a matching Ibis");
    ...
} catch(NestedException e2) {
    System.err.println("Could not create Ibis: " + e2);
    ...
}
\end{verbatim}
\end{quote}
}
Note that the capabilities can be so specific that no matching Ibis
can be found, and therefore \texttt{createIbis()} may throw an exception
indicating this.
\texttt{createIbis()} may also throw a \texttt{NestedException}, indicating
that it tried to create one or more matching Ibises, but could not.
In this case, the exceptions contained inside the \texttt{NestedException}
describe these failures.

\mysubsection{Ibis Capabilities}

Below is a snippet from the RPC example, constructing the required capabilities:
{\small
\begin{quote}
\begin{verbatim}
CapabilitySet capabilities = new CapabilitySet(
    PredefinedCapabilities.CONNECTION_ONE_TO_ONE,
    PredefinedCapabilities.COMMUNICATION_RELIABLE,
    PredefinedCapabilities.SERIALIZATION_OBJECT,
    PredefinedCapabilities.RECEIVE_EXPLICIT,
    PredefinedCapabilities.RECEIVE_AUTO_UPCALLS,
    PredefinedCapabilities.WORLDMODEL_CLOSED);
\end{verbatim}
\end{quote}
}
This states that the selected Ibis must support reliable one-to-one
communication, must support upcalls at the receiver side without the
receiver having to poll for messages, and must also support explicit
receipt of messages at the receiver side.
(We want upcalls so that the server can do other work, and we want
explicit receipt for the client side).
In addition, the selected Ibis must support some form of object
serialization (a string must be sent),
and must support the ``closed'' worldmodel, which means
that all participating Ibises join at the start of the run.

The complete list of predefined capabilities is given in the
\texttt{PredefinedCapabilities} class. The most important ones
are:
\begin{description}
\item[CONNECTION_ONE_TO_ONE]
One-to-one (unicast) communication is supported (if an Ibis implementation
does not support this, you may wonder what it \emph{does} support).
\item[CONNECTION_ONE_TO_MANY]
one-to-many (multicast) communication is supported
(in Ibis terms: a sendport
may connect to multiple receiveports).
\item[CONNECTION_MANY_TO_ONE]
many-to-one communication is supported (in Ibis terms: multiple
sendports may connect to a single receiveport).
\item[COMMUNICATION_FIFO]
messages from a send port are delivered to the receive ports it is
connected to in the order in which they were sent.
\item[CONNECTION_DOWNCALLS]
connection downcalls are supported. This means that the user can
invoke methods to see which connections were lost or created.
\item[CONNECTION_DOWNCALLS]
connection upcalls are supported. This means that an upcall
handler can be installed that is invoked whenever a new connection arrives
or a connection is lost.
\item[COMMUNICATION_NUMBERED]
all messages originating from any send port of a specific port type have
a sequence number. This allows the application to do its own sequencing.
\item[COMMUNICATION_RELIABLE]
reliable communication is supported, that is,
a reliable communication protocol is used.
When not specified, an Ibis implementation may be chosen that does not explicitly
support reliable communication.
\item[RECEIVE_AUTO_UPCALLS]
upcalls are supported and polling for them is not required.
This means that when the user creates a receiveport with an upcall
handler installed, when a message arrives at that receive port, 
this upcall handler is invoked automatically.
\item[RECEIVE_POLL_UPCALLS]
upcalls are supported but polling for them may be needed. When an
Ibis implementation claims that it supports this, it may also do
RECEIVE_AUTO_UPCALLS, but polling does no harm. When an application asks for
this (and not RECEIVE_AUTO_UPCALLS), it must poll.
\item[RECEIVE_EXPLICIT]
explicit receive is supported.
This is the alternative to upcalls for receiving messages.
\item[SERIALIZATION_BYTE]
Only the methods \texttt{readByte()}, \texttt{writeByte()}, \texttt{readArray(byte[])} and \texttt{writeArray(byte[])} are supported.
\item[SERIALIZATION_DATA]
Only \texttt{read()}/\texttt{write()} and \texttt{readArray()}/\texttt{writeArray()} of primitive types are supported.
\item[SERIALIZATION_OBJECT]
Some sort of object serialization is supported.
This requires user-defined
\texttt{writeObject()}/\texttt{readObject()} methods to be symmetrical, that is,
each write in \texttt{writeObject()} must have a corresponding read
in \texttt{readObject()} (and vice versa).
\item[SERIALIZATION_STRICTOBJECT]
Sun serialization (through \texttt{java.io.ObjectOutputStream/InputStream}) is
supported. In some regards, this object serialization implementation is more
forgiving than SERIALIZATION_OBJECT.
\item[WORLDMODEL_OPEN]
Ibises can join and leave the run at any time during the run.
\item[WORLDMODEL_CLOSED]
The number of nodes involved in the run is known in advance and
available from the \texttt{Ibis.totalNrOfIbisesInPool()} method.
\end{description}

\noindent
If a specific implementation of Ibis is required, that can be dealt with too.
There is a property called \texttt{ibis.name}, which can be used to supply the
classname of the required Ibis implementation.

\subsection{Setting up Communication}

Setting up communication consists of several steps:
\begin{itemize}
\item
create a port type;
\item
create a send and a receive port;
\item
set up connections between them.
\end{itemize}

\noindent
The next few subsections discuss each of these steps in turn, but
first we will discuss how to decide which Ibis instance does what.

\mysubsubsection{Which Instance Does What?}

Up until now, we have discussed only matters that all instances of
the Ibis application should do, but now things become different.
One instance of the application may want to send messages, while
another instance may want to receive them.
It may not even be clear which instance is going to do what.
This can of course be solved with program parameters, but Ibis
also provides a so-called registry (of type
\texttt{ibis.ipl.Registry}), which is obtained through the
\texttt{ibis.registry()} method.
Ibis also provides the \texttt{ibis.ipl.IbisIdentifier} class.
The \texttt{ibis.ipl.Ibis.identifier()} method returns such an
Ibis identifier, which identifies this specific Ibis instance.

Using these methods it is possible to decide, in the RPC example,
who is going to be the server by means of an ``election'': the Ibis
registry provides a method \texttt{elect()} which (globally) selects
one of a number of invokers.  For our RPC example this could be done as
follows:

{\small
\begin{quote}
\begin{verbatim}
IbisIdentifier me = ibis.identifier();
Registry registry = ibis.registry();
IbisIdentifier server = registry.elect("Server");
boolean iAmServer = server.equals(me);
IbisIdentifier client = null;
if (iAmServer) {
    client = registry.getElectionResult("Client");
} else {
    client = registry.elect("Client");
}
\end{verbatim}
\end{quote}
}

In our example, one instance of the program is the server and the
other instance is a client.  Of course, the client and the server can also
be different programs altogether.
Note that the server finds out who is the client by looking at an election
result without being a candidate.

The \texttt{RegistryEventHandler}, as discussed in Section
\ref{Creating an Ibis Instance}, can be used to keep track of the number
of Ibis instances currently involved in the run.

\subsubsection{Creating a Port Type}

To be able to create send and receive ports, it is first necessary
to create one or more \emph{port types}.
A port type is an object
of type \texttt{ibis.ipl.PortType}.
Within an Ibis instance,
multiple port types, with different properties, can be created.
The capabilities of a port type are, like the required capabilities
of an Ibis implementation, specified by a \texttt{CapabilitySet} object.
A port type is identified by these capabilities.
The \texttt{Ibis} class contains a method to create a port type,
specified as follows:
{\small
\begin{quote}
\begin{verbatim}
PortType createPortType(CapabilitySet portCapabilities);
\end{verbatim}
\end{quote}
}

\noindent
For our RPC example program, we would create a port type with capabilities
as discussed in Section \ref{Creating an Ibis Instance}:

{\small
\begin{quote}
\begin{verbatim}
CapabilitySet portCapabilities = new CapabilitySet(
    PredefinedCapabilities.CONNECTION_ONE_TO_ONE,
    PredefinedCapabilities.COMMUNICATION_RELIABLE,
    PredefinedCapabilities.SERIALIZATION_OBJECT,
    PredefinedCapabilities.RECEIVE_EXPLICIT,
    PredefinedCapabilities.RECEIVE_AUTO_UPCALLS);
PortType porttype = ibis.createPortType(portCapabilities);
\end{verbatim}
\end{quote}
}
\noindent
In general, the port capabilities should be a subset of the capabilities
specified when creating the Ibis instance. If a capability
is specified that was not specified when creating the Ibis instance,
this may result in an \texttt{IbisConfigurationException}.

\subsubsection{Creating Send and Receive Ports}

The \texttt{PortType} class contains several variants of a method
\texttt{createSendPort()} that creates a send port (of type
\texttt{ibis.ipl.SendPort}) and
also several variants of a method \texttt{createReceivePort()} that
creates a receive port (of type \texttt{ibis.ipl.ReceivePort}).
See the API for an exhaustive list of variants.

In Ibis, receive ports usually have specific names, so that
a send port can set up a connection to a receive port. In contrast,
send ports usually are anonymous, because a receive port cannot
initiate a connection.

For our RPC example, the server will have to create a receive port
to receive a request and a send port to send an answer.
The server is not allowed to block waiting for a request, so it will
want a receive port that enables upcalls.
To do that, the server must first define a class that implements
the \texttt{ibis.ipl.MessageUpcall} interface. This interface contains one
method:

{\small
\begin{quote}
\begin{verbatim}
void upcall(ReadMessage m) throws java.io.IOException;
\end{verbatim}
\end{quote}
}

We will go into the details of a \texttt{ReadMessage} in Section
\ref{Communicating}. For now, we will assume that there is a
class \texttt{RpcUpcall} that implements this interface, and
that the application has a field \texttt{rpcUpcall} of this type.

{\small
\begin{quote}
\begin{verbatim}
try {
    SendPort serverSender = porttype.createSendPort();
    ReceivePort serverReceiver =
        porttype.createReceivePort("server", rpcUpcall);
} catch(java.io.IOException e) {
    ....
}
\end{verbatim}
\end{quote}
}

\noindent
The client will have to create a send port
to send a request and a receive port to receive an answer.
The client is allowed to block waiting for an answer, so it will
want a receive port that enables explicit receipt.
So, the client will create an anonymous server port, and a named
receive port that enables explicit receipt (no upcall handler is supplied):
{\small
\begin{quote}
\begin{verbatim}
try {
    SendPort clientSender = porttype.createSendPort();
    ReceivePort clientReceiver =
        porttype.createReceivePort("client");
} catch(java.io.IOException e) {
    ....
}
\end{verbatim}
\end{quote}
}

\noindent
When a receive port is created, it will not immediately accept connections.
This must be explicitly enabled by
invoking the \texttt{enableConnections()} method.
Incoming connection attempts are kept pending until connections are enabled.
So, the creator of
the receive port can determine when he is ready to accept connections.
If the receive port is configured for upcalls, these must
explicitly be enabled by invoking the \texttt{enableMessageUpcalls()} method.
Again, incoming messages are kept pending until upcalls are enabled.

\subsubsection{Setting Up a Connection}

Now that we have send ports and receive ports, it is time to set up
connections between them.
A connection is initiated by the \texttt{connect()} method of
\texttt{ibis.ipl.SendPort}.
Here is its specification:

{\small
\begin{quote}
\begin{verbatim}
void connect (IbisIdentifier ibis, String name) throws IOException;
\end{verbatim}
\end{quote}
}

\noindent
This version blocks until an accept or deny is received.
An \texttt{IOException} is thrown when the connection could not be established.
There also is a \texttt{connect()} version with a time-out.
So, we need a \texttt{IbisIdentifier} and a name to set up the
connection.
Here is the connection setup code for the server:

{\small
\begin{quote}
\begin{verbatim}
try {
    serverSender.connect(client, "client");
} catch(IOException e) {
    ...
}
\end{verbatim}
\end{quote}
}

\noindent
Our RPC client would set up the following connection:

{\small
\begin{quote}
\begin{verbatim}
try {
    clientSender.connect(server, "server");
} catch(IOException e) {
    ...
}
\end{verbatim}
\end{quote}
}

This completes the connection setup.

Note that a send port can set up connections to more than one
receive port (if the port type supports the \texttt{CONNECTION_ONE_TO_MANY}
capability). Also, multiple send ports can set up
connections to the same receive port (if the port type supports
the \texttt{CONNECTION_MANY_TO_ONE} capability).

\mysubsection{Connection upcalls, connection downcalls}

Sometimes it is useful for an application to know which send ports
are connected to a receive port, and vice versa, or which connections
are being closed.
Ibis implementations may support two different mechanisms for obtaining
this type of information: connection upcalls and connection downcalls.
See Section \ref{Capabilities} for the corresponding capabilities.

When a port type is configured for using connection upcalls,
a receive port may be instantiated with a \texttt{ReceivePortConnectUpcall}
object. This is an interface with two methods:

{\small
\begin{quote}
\begin{verbatim}
boolean gotConnection(ReceivePort me,
                      SendPortIdentifier applicant);
void lostConnection(ReceivePort me,
                    SendPortIdentifier johndoe,
                    Exception reason);
\end{verbatim}
\end{quote}
}
\noindent 
The \texttt{gotConnection()} method gets called when a send port attempts
to connect to the receive port at hand.
An implementation of this method can decide whether
to allow this connection or not by returning \texttt{true} or \texttt{false}.
The \texttt{lostConnection()} method gets called when an existing connection
to the receive port at hand gets lost for some reason.

A send port can be instantiated with a
\texttt{SendPortDisconnectUpcall} object.
This is an interface with a single method:

{\small
\begin{quote}
\begin{verbatim}
void lostConnection(SendPort me,
                    ReceivePortIdentifier johndoe,
                    Exception reason);
\end{verbatim}
\end{quote}
}
\noindent 
This method is called when an existing connection from the send port at
hand gets lost for some reason. Note that there is no \texttt{gotConnection()}
counterpart, because it is always the send port that initiates a connection.

When a port type is configured for using connection downcalls, receive ports
and send ports of this type maintain connection information, and support
methods that allow the user to obtain this information.
A receive port has the following methods:

{\small
\begin{quote}
\begin{verbatim}
SendPortIdentifier[] newConnections();
SendPortIdentifier[] lostConnections();
SendPortIdentifier[] connectedTo();
\end{verbatim}
\end{quote}
}

\texttt{newConnections()} returns the send port identifiers of the connections
that are new since the last call or the start of the program.
\texttt{lostConnections()} returns the send port identifiers of the connections
that were lost since the last call or the start of the program.
\texttt{connectedTo()} returns the send port identifiers of all connections
to this receive port.
A send port has the following methods:

{\small
\begin{quote}
\begin{verbatim}
ReceivePortIdentifier[] newConnections();
ReceivePortIdentifier[] lostConnections();
ReceivePortIdentifier[] connectedTo();
\end{verbatim}
\end{quote}
}
\noindent
which do exactly the same as their receive port counterparts.

\mysubsection{Communicating}

Communication in Ibis
consists of messages, sent from a send port, and received at a
receive port. When a sender wants to send a message, it will first
have to obtain one from the send port. Such a message is of
type \texttt{ibis.ipl.WriteMessage} and is obtained by means of
the \texttt{newMessage()} method of \texttt{SendPort}, which is specified
as follows:

{\small
\begin{quote}
\begin{verbatim}
WriteMessage newMessage() throws IOException;
\end{verbatim}
\end{quote}
}

\noindent
For a given send port, only one message can be alive at any time.
When a new message is requested while a message is alive, the request
is blocked until the live message is finished.

Once a write message is obtained, data can be written to it.
A write message has various methods for the different types of
data that can be written to it. For instance, the
\texttt{writeInt()} method can be used to write an integer value,
and the \texttt{writeObject()} method can be used to write an object.
The kind of data that can be written to the message depends on the
serialization capability specified when the port type was created.
The most general form is object serialization, which supports 
all write methods in a write message.
The \texttt{data} serialization capability does not allow use of the
\texttt{writeObject()} method, but does allow the use of all other write
methods. The \texttt{byte} serialization capability only allows use
of the \texttt{writeByte()} method and the \texttt{writeArray(byte[])}
method.

Once there is a considerable amount of data in the message, Ibis
can be given a hint to start sending, using the \texttt{send()}
method. This hint is not required, however. When the message is
complete, the message can be sent out by invoking the
\texttt{finish()} method.

Our client in the RPC example could have the following:
{\small
\begin{quote}
\begin{verbatim}
int obtainLength(String s) throws IOException {
    WriteMessage w = clientSender.newMessage();
    w.writeObject(s);
    w.finish();
    ...
}
\end{verbatim}
\end{quote}
}

\noindent
At the receiving side, a message can be received in two ways,
depending on how the receive port was created: either by means of an
upcall, or by means of an explicit receive. For each write method
in the \texttt{WriteMessage} type, there is a corresponding read method in
the \texttt{ReadMessage} type. For a given receive port, only one message can
be alive at any time. A read message is alive until it is
finished (by a \texttt{finish()} call), or the upcall returns.

Now, let us present some more code of our RPC example, this time
from the server:

{\small
\begin{quote}
\begin{verbatim}
public void upcall(ReadMessage m) throws IOException {
    String s = "";
    try {
	s = (String) m.readObject();
    } catch(ClassNotFoundException e) {
        ...
    }
    int len = s.length();
    m.finish();
    WriteMessage w = serverSender.newMessage();
    w.writeInt(len);
    w.finish();
}
\end{verbatim}
\end{quote}
}

\noindent
Note that the read message is finished before replying to the
request. To prevent deadlocks, upcalls are not allowed to block
(call \texttt{Thread.wait()}) or access the network (write a message or
read another message) as long as a read message is active.

Now, we can also finish the \texttt{obtainLength()} method of the client:
{\small
\begin{quote}
\begin{verbatim}
    ...
    ReadMessage r = clientReceiver.receive();
    int len = r.readInt();
    r.finish();
    return len;
}
\end{verbatim}
\end{quote}
}

\subsection{Finishing up}

Closing of a connection is initiated by closing a send port
by means of the \texttt{close()} method. The \texttt{ReceivePort}
class also has a \texttt{close()} method, but this method blocks
until all send ports that have a connection to it are closed.
So, send ports have to be closed first.

Our RPC client will do the following:

{\small
\begin{quote}
\begin{verbatim}
clientSender.close();
clientReceiver.close();
\end{verbatim}
\end{quote}
}
and the code of the server should be clear by now.

Ibis itself must also be ended. Both our client and our server
should invoke the \texttt{Ibis.end()} method:
{\small
\begin{quote}
\begin{verbatim}
ibis.end();
\end{verbatim}
\end{quote}
}

As of Java 1.3, it is also possible to add a so-called shutdown hook.
This could be done right after the Ibis instance is created:
{\small
\begin{quote}
\begin{verbatim}
Runtime.getRuntime().addShutdownHook(new Thread() {
    public void run() {
        try {
            ibis.end();
        } catch (IOException e) {
        }
    }
});
\end{verbatim}
\end{quote}
}
\noindent
This shutdown hook gets invoked when the program terminates, and
forcibly closes all ports.

XXXXXXXXXXXXXXXXXXXXXXXXXXXXXXXXXXX

\mysubsection{Ibis utilities}

The \texttt{ibis.util} package contains several utilities that may be
useful for Ibis applications.
The \texttt{ibis.util.Stats} utility contains methods for computing
the mean and standard deviation of an array of numbers.
A timer utility is provided in \texttt{ibis.util.Timer}.
See the Ibis API for other utilities. Most of these are used in
Ibis implementations, but may have other uses.

\mysubsection{Avoiding deadlocks}

As with most communication layers, it is quite easy to write code that
deadlocks with Ibis. For example, if you have two Ibis instances that
are writing large amounts of data to each other, and there are no
readers for this data active, this will almost certainly result in a deadlock,
because network buffers will fill up, causing the senders to block.
Such a deadlock can be avoided by having a separate reader thread,
or by installing an upcall handler for the incoming message.

Another common source of deadlocks is if you have a port type that
specifies the \texttt{CONNECTION_MANY_TO_ONE} as well as the
\texttt{CONNECTION_ONE_TO_MANY} capability.
Multiple hosts doing simultaneous multicasts
is a well-known source of deadlocks, because most systems do not
implement a functioning flow-control for these cases.

\mysection{Compiling and Running an Ibis Application}

Before running an Ibis application it must be compiled.  Using
\emph{ant}, this is quite easy. Assuming that the environment variable
\texttt{IBIS\_HOME} reflects the location of your Ibis installation,
here is a \texttt{build.xml} file
for our example program:

{\small
\begin{quote}
\begin{verbatim}
<project
    name="client-server"
    default="build"
    basedir=".">

    <description>
    Ibis application build.
    </description>

    <property environment="env" />
    <property name="ibis"   value="${env.IBIS_HOME}" />

    <property name="build"  location="build"/>

    <import file="${ibis}/build-files/apps/build-ibis-app.xml"/>
</project>
\end{verbatim}
\end{quote}
}

Now, invoking \emph{ant} compiles the application, leaving the class files
in a directory called \texttt{build}.

If, for some reason, it is not convenient to use \emph{ant} to compile
your application, or you have only class files or jar files available
for parts of your application, it is also possible to first compile
your application to class files or jar files, and then process those
using the \emph{ibisc} script. This script can be found in the Ibis
bin directory. It takes either directories, class files, or jar files
as parameter, and processes those, possibly rewriting them. In case
of a directory, all class files and jar files in that directory or
its subdirectories are processed.

\mysubsection{The Ibis Registry}

Most Ibis implementations depend on a registry server for providing
information about a particular run, such as finding Ibis instances
participating in the run, elections, et cetera.
The Ibis registry server collects this information for multiple Ibis
runs, even simultaneous ones. It does so by associating a user-supplied
identifier with each Ibis run. Each Ibis instance announces its
presence to the registry server, using this identifier, so that the
registry server can determine to which Ibis run this Ibis instance belongs.
The registry server then notifies the other Ibis instances of this run that
a new instance has joined the run, including some identification of
this instance.

If you tell Ibis that the registry server location is a machine that also
participates in the run itself, Ibis will automatically try to start
a registry server. How you can specify this is explained in the next section.
If you want to run the registry server on a seperate host, one that is not
behind a firewall, for instance, you have to start the registry server by
hand.

The Ibis registry server is started with the \texttt{ibis-registry-server}
script which lives in the Ibis \texttt{bin} directory.
Before starting an Ibis application,
you need to have a registry server running on a machine that is accessible
from all nodes participating in the Ibis run.
The registry server expects the Ibis instances to connect to a
socket that it creates when it starts up.
The port number of this socket can be specified using the
\texttt{-port} command line option to the \texttt{ibis-registry-server} script.

\mysubsection{Running an Ibis Application}

An Ibis instance is started with the \texttt{ibis-run} script which
lives in the Ibis \texttt{bin} directory.  This \texttt{ibis-run}
script is called as follows:
\begin{center}
\texttt{ibis-run} \emph{java-flags class params}
\end{center}
The \texttt{ibis-run} script is just a small script that adds the jar files
from the Ibis lib directory to your classpath and then starts Java.

XXXXXXXXXXXXXXXXXXXXXXXXXXXXXXXXXXXXXXXXXXXXXXXXXXXXXXXXXXXXXXXXXXXXXXXXXXXXXX

The \texttt{ibis-run} script uses these parameters to set the following
system properties (see also Section \ref{PoolInfo}):
\begin{description}
\item{\texttt{ibis.pool.host\_number}}
the rank number of this Ibis instance.
\item{\texttt{ibis.pool.total\_hosts}}
the total number of Ibis instances.
\item{\texttt{ibis.name\_server.pool}}
identifies the run to the nameserver.
\item{\texttt{ibis.name\_server.port}}
the nameserver port.
\item{\texttt{ibis.name\_server.host}}
the nameserver hostname.
\end{description}

In addition, the following system properties can be specified
(as \emph{javaflags}):
\begin{description}
\item{ibis.pool.cluster}
specifies a cluster name. If not specified,
``unknown'' is used.
See Section \ref{Satin job scheduling} for a possible need for cluster
names.
\item{ibis.pool.server.port}
specifies the port number on which the
\texttt{PoolInfoServer} is listening.
\end{description}

The Ibis distribution also provides \emph{grun}, which is a tool for
running Ibis applications on a Globus-based grid. See the
\texttt{grun/doc} subdirectory for more information.

\mysubsection{Running the example}
In order to run the example, we first have to compile it.
Please go to the \emph{ibis-example} directory and type:

\noindent
{\small
\begin{verbatim}
$ ant
\end{verbatim}
}
\noindent
After a couple of seconds, the example should be compiled.
You can run the example both with and without the \texttt{ibis-run} script.
Because we run the example on a single machine, we do not need to start a seperate nameserver.
To run the application, we first need to start two shells.
Then, in both shells type:

\noindent
{\small
\begin{verbatim}
$ $IBIS_HOME/bin/ibis-run \
    -nhosts 2 -ns localhost ibis.ipl.apps.example.Example
\end{verbatim}
}
\noindent
Now, it should run and print:

\noindent
{\small
\begin{verbatim}
Test succeeded!
\end{verbatim}
}
\noindent
If you don't use the \texttt{ibis-run} script, you have to set several properties.
In both shells, type:

\noindent
{\small
\begin{verbatim}
$ java \
    -cp $IBIS_HOME/lib/ibis.jar:$IBIS_HOME/3rdparty/log4j-1.2.9.jar:build \
    -Dibis.pool.total_hosts=2 -Dibis.registry.host=localhost \
    -Dibis.registry.pool=bla \
    ibis.ipl.apps.example.Example
\end{verbatim}
}
\noindent
The \texttt{ibis.name\_server.pool} value can be any random string.
It identifies your run.
This is done because one nameserver can serve multiple
runs. For this test, we need to provide Ibis with the total number of
CPUs in the run, because it is a closed-world test. Ibis waits until
both processors have joined the computation.
The \texttt{ibis.name\_server.host} property should
be set to the machine you run the nameserver on.  In this case, we use
\texttt{localhost}.
Because we also run the application on \texttt{localhost}, Ibis will
automatically start a nameserver. If you provide a hostname where the
Ibis application does not run, you will have to start a nameserver
yourself.

\mysection{Further Reading}

The Ibis web page
{\html {\htmladdnormallink{http://www.cs.vu.nl/ibis/publications.html} {http://www.cs.vu.nl/ibis/publications.html}}}
{\latex {http://www.cs.vu.nl/ibis/publications.html}}
contains links to various Ibis papers.
The best starting point might be \\
{\html {\htmladdnormallink{http://www.cs.vu.nl/ibis/papers/nieuwpoort_cpe_05.pdf} {http://www.cs.vu.nl/ibis/papers/nieuwpoort\_cpe\_05.pdf}}}
{\latex {http://www.cs.vu.nl/ibis/papers/nieuwpoort\_cpe\_05.pdf}}, which gives a high-level description of the stucture of the Ibis system.
It also gives some low-level and high-level benchmarks of the different Ibis implementations.

The \emph{docs/api} subdirectory of the Ibis installation provides
documentation for each class and method in the Ibis API (point your favorite
HTML viewer to docs/api/index.html in the Ibis installation).
The Ibis API is also available
\html{\htmladdnormallink{on-line}{http://www.cs.vu.nl/ibis/api/index.html}.}
\latex{on-line at http://www.cs.vu.nl/ibis/api/index.html.}

\end{document}
