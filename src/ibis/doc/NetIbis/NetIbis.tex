\documentclass[11pt]{book}


%% Packages

% Language
\usepackage[english]{babel}

% Page dimensions
\usepackage{fullpage}

% Fonts
\usepackage{palatino}

% Images
\usepackage{graphicx}

% Fx objects
\usepackage{shadow}
\usepackage{boxedminipage}
\usepackage{ulem}

% Special text environments
\usepackage{fancyvrb}
\usepackage[obeyspaces]{url}

% Programming tools
\usepackage{xspace}
\usepackage{ifthen}
\usepackage{calc}

% Gray levels management
\usepackage{color}


%% Setup of packages

% Ulem
\normalem

% FancyVrb
\fvset{numbers=left}
\fvset{numbersep=2pt}
%\fvset{obeytabs=true}
\fvset{obeytabs=false}
\fvset{resetmargins=true}
\fvset{frame=lines}
\fvset{framerule=0mm}
\fvset{framesep=1mm}


% ...... Background grays ............................................
\definecolor{light}{gray}{.85}
\definecolor{vlight}{gray}{.95}

% ...... Aux. objects ................................................
\newcounter{mylstline}
\newlength{\mylstlength}

% ...... Listing logic ...............................................
\makeatletter
\def\DefineVerbatimEnvironmentEx#1#2#3#4{%
  \@namedef{#1}{\renewcommand{\FancyVerbFormatLine}{#4}\FV@Environment{#3}{#2}}%
  \@namedef{end#1}{\@nameuse{FVE@#2}}%
  \@namedef{#1*}{\renewcommand{\FancyVerbFormatLine}{#4}\FV@Environment{#3,showspaces}{#2}}%
  \@namedef{end#1*}{\@nameuse{FVE@#2}}}
\makeatother
%\renewcommand{\FancyVerbFormatLine}[1]%
\newcommand{\FancyVerbFormatLineRegular}[1]%
{%
  \footnotesize%
  {%
    \setcounter{mylstline}{\value{FancyVerbLine}/2*2}%
    \setlength{\mylstlength}{\linewidth - 1em - 2pt}%
    \ifthenelse%
    {%
      \value{mylstline} = \value{FancyVerbLine}%
    }%
    {%
      \makebox[0pt][l]%
      {%
        \colorbox{light}%
        {%
          \rule[-0.5ex]{0cm}{.9\baselineskip}%
          \hspace{\mylstlength}%
        }%
      }%
    }%
    {%
      \makebox[0pt][l]%
      {%
        \colorbox{vlight}%
        {%
          \rule[-0.5ex]{0cm}{.9\baselineskip}%
          \hspace{\mylstlength}%
        }%
      }%
      \makebox[0pt][l]%
      {%
        \begin{picture}(0,0)(-5,-2)\put(0,0){\circle{4}}\end{picture}%
      }%
      \makebox[0pt][l]%
      {%
        \color{white}%
        \begin{picture}(0,0)(-5,-2)%
          \put(0,0){\circle*{2}}%
        \end{picture}%
      }%
      \makebox[0pt][l]%
      {%
        \hspace{\mylstlength}%
        \begin{picture}(0,0)(0,-2)%
           \put(0,0){\circle{4}}%
        \end{picture}%
      }%
      \makebox[0pt][l]%
      {%
        \hspace{\mylstlength}%
        \color{white}%
        \begin{picture}(0,0)(0,-2)%
          \put(0,0){\circle*{2}}%
        \end{picture}%
      }%
    }%
%
    \hspace{1cm}#1%
  }%
}

\newlength{\deffvrbindent}
\setlength{\deffvrbindent}{\parindent}


\newcommand{\FancyVerbFormatLineMini}[1]%
{%
  \footnotesize%
  \setlength{\baselineskip}{1.5ex}
  {%
    \setcounter{mylstline}{\value{FancyVerbLine}/2*2}%
    \setlength{\mylstlength}{\linewidth - 1em - 2pt}%
    \ifthenelse%
    {%
      \value{mylstline} = \value{FancyVerbLine}%
    }%
    {%
      \makebox[0pt][l]%
      {%
        \colorbox{light}%
        {%
          \rule[-0.25ex]{0cm}{\baselineskip}%
          \hspace{\mylstlength}%
        }%
      }%
    }%
    {%
      \makebox[0pt][l]%
      {%
        \colorbox{vlight}%
        {%
          \rule[-0.25ex]{0cm}{\baselineskip}%
          \hspace{\mylstlength}%
        }%
      }%
      \makebox[0pt][l]%
      {%
        \begin{picture}(0,0)(-5,-2)\put(0,0){\circle{4}}\end{picture}%
      }%
      \makebox[0pt][l]%
      {%
        \color{white}%
        \begin{picture}(0,0)(-5,-2)%
          \put(0,0){\circle*{2}}%
        \end{picture}%
      }%
      \makebox[0pt][l]%
      {%
        \hspace{\mylstlength}%
        \begin{picture}(0,0)(0,-2)%
           \put(0,0){\circle{4}}%
        \end{picture}%
      }%
      \makebox[0pt][l]%
      {%
        \hspace{\mylstlength}%
        \color{white}%
        \begin{picture}(0,0)(0,-2)%
          \put(0,0){\circle*{2}}%
        \end{picture}%
      }%
    }%
%
    %\hspace{1cm}#1%
    \hspace{\deffvrbindent}#1%
  }%
}

\renewcommand{\FancyVerbFormatLine}{\FancyVerbFormatLineRegular}
\DefineVerbatimEnvironmentEx{Miniverb}{Verbatim}{%
baselinestretch=0}{%
\FancyVerbFormatLineMini%
}

%%
\let\bfseriesaux=\bfseries \def\bfseries{\sffamily\bfseriesaux}
\sloppy


%% Definitions

\newcommand{\newdoublepage}{\newpage\cleardoublepage}

\def\sep{\\[0.5ex]}

\def\Java{\textsc{Java}\xspace}
\def\Ibis{\textsc{Ibis}\xspace}
\def\NetIbis{\textsc{NetIbis}\xspace}
\def\TcpIbis{\textsc{TcpIbis}\xspace}
\def\PandaIbis{\textsc{PandaIbis}\xspace}

\def\ibis{\texttt{Ibis}\xspace}
\def\netibis{\texttt{NetIbis}\xspace}

\def\jd{\textsc{JavaDoc}\xspace}
\def\ins{\textsc{Name~Server}\xspace}

%% First page information
\title{The NetIbis Manual}
\author{Olivier Aumage}
\date{\today}

\begin{document}

\frontmatter
\maketitle

\tableofcontents
\listoffigures
\listoftables
\mainmatter
%______________________________________________________________________

%%%%
\chapter{User's guide}
\label{cha:users-guide}

This chapter presents \NetIbis from the user --- that is from the
application programmer --- point of view. Section~\ref{sec:concepts}
presents the various entities and notions needed to understand the
basic use of \NetIbis. Section~\ref{sec:configuration} describe the
configuration rules. Section~\ref{sec:progr-interf} details the \NetIbis
application programming interface.

%%%
\section{Concepts}
\label{sec:concepts}

Several entities and notions are involved during a distributed
computing session with \NetIbis. We now review the ones that are
visible from the user application.

%%
\subsection{Ibis/NetIbis}
\label{sec:ibisnetibis}

\NetIbis is an implementation of the \Ibis Programming Layer (IPL) as
\TcpIbis and \PandaIbis are. \netibis is also a subclass of the \ibis
class written in \textsc{Java}. Instantiating the \netibis class both
initializes the core part of \NetIbis and provide an entry-point to
the \NetIbis programming interface. Even if it might work in some
cases, it is considered \emph{unsafe} to create more than one \netibis
instance --- and more generally, more than one \ibis instance --- in
the same process, during the lifetime of that process.

%%
\subsection{Communication Ports}
\label{sec:communication-ports}

\Ibis uses the concept of communication port as an entry-point for
outgoing communication (send port) or incoming communication (receive
port). A send port can be connected to one or more receive ports. A
receive port may accept incoming connections from one or more send
ports.

%
\subsubsection{Connections}
\label{sec:connections}

In the Ibis context a connection is a unidirectional FIFO-ordered
virtual communication link between two (possibly not distinct)
processes. Connections are isolated from each other unless the
underlying communication software/hardware prevents this.


\subsubsection{Send Ports}
\label{sec:send-ports}

Sending a message through a send port actually sends the message to
each of the peer receive ports of this send port. Thus, connecting a
send port to more than one receive port is a way to achieve multi-cast
communication, provided that a multi-casting driver has been included
in the port's driver tree (see \ref{sec:driver-trees}).


\subsubsection{Receive Ports}
\label{sec:receive-ports}

This behavior is of course not completely symmetrical on the receive
side. In the case that a receive port accepted more than one incoming
connection, the incoming messages over these connections are
serialized. It is not possible to extract a new message while the
previous one has not been completely extracted.

\subsubsection{Port Type}
\label{sec:port-type}

The send and receive port may be configured, either in a configuration
file or dynamically at the creation of the port. The set of
configuration properties of a port is called a \textit{port type}. The
port type of a send port and the port type of a receive port must
exactly match in order to be connected to each other.


%%%
\section{Configuration}
\label{sec:configuration}

The \NetIbis implementation of \Ibis has been designed to make the
send port/receive port communication paths completely configurable,
according to the actual needs (or non-needs) of the application. These
communication paths are built using one or more \textit{drivers}
organized as a \textit{driver tree}.

\subsection{The Driver Trees}
\label{sec:driver-trees}

Each driver provide one \textit{added value}, for instance a
\textit{filtering} added value like the ``serialization of Java
objects into Java native types'' or a \textit{networking} added value
like the ``transmission of blocks of bytes over GM''. Thus, the
\NetIbis drivers should be seen has basic building blocks for
communication stacks. The filtering drivers are listed in
Table~\ref{tab:filter-drivers} and the networking drivers are listed
in Table~\ref{tab:net-drivers}.

\begin{table}[htbp]
  \centering
  \begin{tabular}{p{2cm}p{10cm}}
    \hline
    \textbf{Name}    & \textbf{Added value} \sep
    \texttt{id}      & \textit{Identity} driver, does nothing\\
    \texttt{rel}     & \textit{Reliability} driver\\
    \texttt{bytes}   & ``Native types'' to ``bytes'' conversion\sep
    \texttt{gen}     & Generic multi-connection splitter/poller\\
    \texttt{multi}   & Multi-protocol driver\sep
    \texttt{s\_ibis} & \Ibis \textit{serialization} driver\\
    \texttt{s\_sun}  & \textsc{Sun} \emph{serialization} driver\\
    \hline
  \end{tabular}
  \caption{Filtering Drivers}
  \label{tab:filter-drivers}
\end{table}

\begin{table}[htbp]
  \centering
  \begin{tabular}{p{2cm}p{10cm}}
    \hline
    \textbf{Name}    & \textbf{Added value} \sep
    \texttt{def}        & Default driver (data goes through the service connection)\sep
    \texttt{pipe}       & Pipe-based loop-back driver (for intra process communication)\\
    \texttt{gm}         & Driver Myricom GM networks\\
    \texttt{udp}        & UDP driver (\emph{unreliable}) \sep
    \texttt{tcp}        & TCP+DataStream\\
    \texttt{tcp\_blk}   & TCP with block transfers\\
    \texttt{tcp\_plain} & Pure TCP driver\\
    \hline
  \end{tabular}
  \caption{Networking Drivers}
  \label{tab:net-drivers}
\end{table}

Each filtering driver may have one or more sub-drivers. As a result,
the general case of a configured send or receive port communication
stack is a tree of zero or more filtering drivers with each leaf being
a networking driver. The \textit{multi-protocol} driver is an example
of a driver owning one or more sub-drivers, which is its main purpose.

\subsection{The Configuration File}
\label{sec:configuration-file}

The communication stack of a port can be specified as a set of
specially formatted properties. Each property is a \texttt{(name,
  value)} pair.

These properties may be written into a configuration file and/or set
at run-time when the port type is being created. Both means are
equivalent with the exception that any run-time specified property
takes precedence over the same property in a configuration file. By
decreasing order of priority, the actual configuration file used is
chosen as follows.

\begin{itemize}
\item The value of the \texttt{netibis.config.filename} system
  property or the \texttt{ibis.net.config.file} system property.
\item The file named \texttt{net\_port\_type\_default.txt} in the
  \emph{current} directory.
\item The file named \texttt{net\_port\_type\_default.txt} in the
  \emph{home} directory.
\item No configuration file.
\end{itemize}

The format of the configuration file is line oriented. Each line may
contain one of the following contents.

\begin{itemize}
\item A property specified as: \texttt{<name>=<value>}.
\item A comment beginning with \texttt{\#}.
\item A blank line.
\end{itemize}

The \emph{name} of the
property is either a single word or a string conforming to the
following format.

\begin{Miniverb}

  [/][context1[#subcontext1][/cntxt2[#subcntxt2].../cntxtN[#subcntxtN]]]:word

\end{Miniverb}

The square brackets indicate optional parts. The \emph{context} words and the
\emph{sub-context}\label{subcontext} words specify in which restricted context the property
named \texttt{word} will be available. The \emph{context} word is
(usually) the name of a driver. The optional \emph{sub-context} word is
used to \emph{narrow} the context of the property.

The part of the property name before the colon should be seen as a
kind of \emph{path} into the driver tree. If the leading slash is
used, the path is an \emph{absolute} path, otherwise it is a relative
path, just like with usual \textsc{Unix} file-systems paths. A single
slash corresponds to the port itself. Here are a few examples of
property settings.

A single word property (\texttt{Log}). This configuration line turns
the low-level trace debugging routines on in all the \NetIbis drivers.
\begin{Miniverb}

  Log=true

\end{Miniverb}

The following configuration lines specifies the \texttt{Driver}
property for the port. This is the usual way to select the topmost
driver. Here, we select the ``Ibis serialization'' filtering driver.
\begin{Miniverb}

  /:Driver=s_ibis

\end{Miniverb}

We now select the sub-driver for the \texttt{s\_ibis} driver, here the
``Native-type to byte conversion'' driver.
\begin{Miniverb}

  /s_ibis:Driver=bytes

\end{Miniverb}

We now select the sub-driver for the \texttt{bytes} driver, here the
``Multi-protocol'' driver.
\begin{Miniverb}

  /s_ibis/bytes:Driver=multi

\end{Miniverb}

Then, we set some properties for the \texttt{multi} driver. We select
a sub-driver, here the generic multi-cast driver. We also select a
multi-protocol scheme plug-in, here the \texttt{random} plug-in which
randomly selects a driver subtree out of two (the subtree corresponding
to the `\texttt{a}' sub-context and the one corresponding to the
`\texttt{b}' sub-context) for each new network connection.
\begin{Miniverb}

  /s_ibis/bytes/multi:Driver=gen
  /s_ibis/bytes/multi:Plugin=random

\end{Miniverb}

We concludes this example configuration by selecting two networking
drivers, one for each subtree generated by the \texttt{multi} driver.
Here we select the \texttt{tcp\_blk} networking driver for the
`\texttt{a}' branch and the \texttt{gm} networking driver for the
`\texttt{b}' branch.
\begin{Miniverb}

  /s_ibis/bytes/multi:Plugin=abc
  /s_ibis/bytes/multi#net_a/gen:Driver=tcp_blk
  /s_ibis/bytes/multi#net_b/gen:Driver=gm

\end{Miniverb}

Suppose we want to include the dummy ``identity'' driver in-between
the \texttt{gen} driver and the networking drivers on both sub-trees.
We just need to replace the two previous lines with the following
ones.
\begin{Miniverb}

  /s_ibis/bytes/multi:Plugin=abc
  /s_ibis/bytes/multi/gen:Driver=id
  /s_ibis/bytes/multi#net_a/gen/id:Driver=tcp_blk
  /s_ibis/bytes/multi#net_b/gen/id:Driver=gm

\end{Miniverb}
There is no sub-context string on the first of the three lines, which
signify that the property value will be used as a default for the two
subtrees. The somewhat general rule is that the more narrow the
context of the property is, the more priority it gets.

A side effect of this rule is that a property specified with an
absolute path always gets the priority over a property with a relative
path. Therefore, any attempt to simplify the configuration should be
accompanied with a careful checking of potential collision.

Another side effect is that a property specified with a single word
always gets the lowest priority. This can be used to set a general
default property than can later be overridden locally.

If we suppose that our example configuration does not contain any
other setting, the configuration can therefore be simplified as
follows.
\begin{Miniverb}

  Log=true
  Driver=s_ibis
  s_ibis:Drivers=bytes
  bytes:Drivers=multi
  multi:Driver=gen
  gen:Driver=id
  multi:Plugin=abc
  multi#net_a/gen/id:Driver=tcp_blk
  multi#net_b/gen/id:Driver=gm

\end{Miniverb}

%%%
\section{Programming Interface}
\label{sec:progr-interf}
Let us now have a look at how to use \NetIbis in distributed
application programming. We first explain the major steps required in
setting-up the application for a communication session \emph{before}
entering the application core computing part and in cleaning the
communication session \emph{after} it.

%%
\subsection{NetIbis Instance}
\label{sec:netibis-instance}

As explained in section~\ref{sec:ibisnetibis}, the main entry point of
the \NetIbis application programming interface (API) is the \netibis
class. The first step an application should do in order to use
\NetIbis is therefore to create an instance of the \netibis class.

%
\subsubsection{Initialization}
\label{sec:initialization}

One way to create a \netibis instance is to use the
\texttt{createIbis} static method of the \texttt{Ibis} class. Here is
an example:

\begin{Miniverb}

  Ibis ibisInstance = null;

  try {
    ibisInstance = Ibis.createIbis(null);
  } catch (IbisException e) {
    throw new Error(e);
  }

\end{Miniverb}

The first and only argument of the \texttt{createIbis} method may be
used to specify a handler to be notified of configuration size
changes. See \jd for more details. The actual \Ibis implementation to
instantiate may be selected through the \texttt{ibis.name}
command-line system property. Set this property to \texttt{"net"} in
order to select \NetIbis as the \Ibis implementation to use.

%
\subsubsection{Clean-up}
\label{sec:clean-up}

At the end of the computing session, the \netibis instance should be
cleaned using its \texttt{end} method, after any other \NetIbis object
has been disposed of. Among other things, this ensures that the \ins
is properly made aware that the current node is about to leave.

%%
\subsection{Ports}
\label{sec:ports}

Once the \netibis instance is ready, it can be used for creating port
types. These port types may in turn be used in creating send and
receive ports conforming to these port types.

%
\subsubsection{Initialization}
\label{sec:initialization-2}

Creating a \emph{port type} is done using the \texttt{createPortType}
method of the \netibis class instance. This method requires a name and
a set of properties. The set of properties will be completed with the
contents of the \NetIbis configuration file if used. The port name
visibility range is the whole session. Therefore, if a process creates
a port type \texttt{a} with a set of properties \texttt{A} and next
another process creates another port type named \texttt{a} with a set
of properties \texttt{B}, the sets \texttt{A} and \texttt{B} must
exactly match. they must have the same number of properties with the
same name and the same value. This requirement is necessary to ensure
that the two ports with the same port type will be compatible.

Creating send and receive ports involve the use of the various
\texttt{createSendPort} and \texttt{createReceivePort} methods of the
port type object. The creation of a send port only requires an
optional name (useful mainly for debugging) and an optional send port
replacer~\footnote{TO-DO: Give a definition of a replacer.}.

The creation of a receive port requires a \emph{mandatory}
\texttt{name}, an optional \texttt{up-call handler} for asynchronous
message receiving and an optional \texttt{connection up-call} for
incoming connection notification\footnote{The incoming connection
  notification feature is actually not supported in the current
  implementation of \NetIbis}. The \texttt{name} argument is the name
that will be used by other nodes to refer to this receive port when
later requesting the \emph{receive port identifier} to the name server.

%
\subsubsection{Incoming Connection Setup}
\label{sec:incom-conn-setup}

Accepting a new incoming connection on a receive port does not require
any special step. In should only be insured that the receive port is
currently \emph{allowing} incoming connections.

Incoming connections are initially disabled. They can be enabled using
the \texttt{enableConnections} method of the receive port and can
later be disabled again using the \texttt{disableConnections} method.
In \NetIbis, both methods can be safely called multiple times in a
row, but two or more calls do not stack. Hence, a single call to
\texttt{enableConnections} always enable the incoming connections, no
matter how many times \texttt{disableConnections} was called.
Similarly, a single call to \texttt{disableConnections} always
disables incoming connections, no matter how many times
\texttt{enableConnections} was called.


\paragraph{Note:}
\label{sec:note}

The \texttt{disableConnections} method may block if a pending
incoming connection is currently being processed.

%
\subsubsection{Outgoing Connection Setup}
\label{sec:outg-conn-setup}

In order to connect a send port to a receive port, the \emph{receive
  port identifier} is necessary. This receive port identifier can be
obtain using by requesting a lookup to the session's name server using
the receive port name.

Actually connecting a send port to a receive port is done using the
send port \texttt{connect} method. This method requires the targeted
receive port identifier and an optional millisecond
timeout\footnote{The timeout version of the \texttt{connect} method is
  currently not implemented in \NetIbis}. This method does not return
anything but may throw an exception if the connection request fails.

%
\subsubsection{Communication}
\label{sec:communication-1}

The \Ibis communication model is a \emph{message passing} model.
Therefor, communication between nodes involves the construction (on
the sending side) and extraction (on the receiving side) of
symmetrical delimited series of pieces of data.

\paragraph{Sending A Message}
\label{sec:sending-message}

The message construction sequence begins with a call to the
\texttt{newMessage} method of the send port. This method does not require
any argument in contrast with some other message passing environments,
because the destination(s) of the message is (are) the receive port(s)
to which the send port is currently connected. It should be noted that
with the current version of \NetIbis, attempting to send a message
through a send port that does not have any outgoing connections will
cause the \texttt{newMessage} method to throw an exception.

If successful, the call to the \texttt{newMessage} method returns a
\texttt{WriteMessage} object which will be use to actually append
pieces of data to the message. In the case of the \NetIbis
implementation, the \texttt{WriteMessage} object is actually the send
port itself\footnote{\texttt{WriteMessage} is an interface.} but other
\Ibis implementations might use different approaches.

The application appends data to the message using the appropriates
\texttt{write*} methods of the \texttt{WriteMessage} object (see \jd
for the details). The \NetIbis implementation may or may not
immediately send each piece of data. The application should not rely
on the specific behavior of an implementation. In particular, the
application should not modify any piece of data already appended to
the message while this message is still under construction.

The message construction ends with a call to the \texttt{send} method
of the \texttt{WriteMessage} object followed by a call to the
\texttt{finish} method of the same object. The \texttt{send} call
maybe used by some implementations that fully bufferize the whole
message to know that this message is complete and can be sent. The
\texttt{finish} call is used to wait for the send request completion.
Once the \texttt{finish} call returns, the message has been sent.
This method also `discard' the \texttt{WriteMessage} object.

\paragraph{Receiving A Message}
\label{sec:receiving-message}

There are two main ways to receive a message, either with an
\emph{explicit receive} operation or with an \emph{up-call}. The former
one is synchronous while the later one is asynchronous.

An synchronous message receive can be done using the \texttt{receive}
method or the \texttt{poll} method of the receive port. The
\texttt{receive} method is \emph{blocking}. It will not return until
an incoming message is ready to extract. The \texttt{poll} method is
the corresponding \emph{non-blocking} version. Both methods return a
\texttt{ReadMessage} object in case of success. In the case of
\NetIbis, the \texttt{ReadMessage} object is actually the receive port
itself. The \texttt{poll} method returns \texttt{null} if there is not
any message to extract immediately.

It should be noted that when a \texttt{poll} or \texttt{receive} call
successfully returns, the incoming message \emph{must} be extracted
before doing a new \texttt{poll} or \texttt{receive} call again.
Multiple \texttt{poll} calls in a row are only allowed while the
polling requests are unsuccessful\footnote{There is of course no
  problem with \texttt{receive} as this method only returns when a
  message is available}.

The asynchronous message receive support requires the application to
register an \emph{up-call handler} at the receive port creation time,
by providing an object that implements the \texttt{ibis.ipl.Upcall}
interface. This interface only has a single method called
\texttt{upcall}. The \texttt{upcall} method gets a
\texttt{ReadMessage} object as its first and only argument.

Whether obtained through a synchronous receive or an up-call handler,
the \texttt{ReadMessage} object can be used to extract the
corresponding incoming message. This is done using the set of
\texttt{read} methods of the \texttt{ReadMessage} object. The series
of extraction calls to receive the pieces of data constituting the
message must \emph{exactly match} the series of construction calls
that was used to send this message. By ``exactly match'', we mean that
the order and the number of calls must be the same on each side of the
connection\footnote{Or the connection\textbf{s} interface the send
  port is connected to multiple receive ports}. In particular, sending
an array or array slice as one single piece and receiving it as
several smaller pieces or single array elements (or the other way
round) will result in unspecified behavior. It is also unspecified to
send (receive) a native type array as an object while receiving
(sending) it as a plain native type array (or array slice).

Once the extraction is complete, the \texttt{ReadMessage} object is
discarded using its \texttt{finish} method. In the special case of an
up-call handler, simply returning from the up-call handler is (almost) equivalent
to calling \texttt{finish}, and the \texttt{finish} call is therefore
optional. The only difference between a \texttt{finish} and up-call
handler return is that the application is allowed to ``keep'' using
the up-call thread after calling \texttt{finish}. In that case, an
subsequent incoming message will result in executing the corresponding
up-call on a new fresh thread. It should be noted that no new up-call
may be generated while a former message extraction has no been
completed by either calling \texttt{finish} or leaving the up-call
handler.

The actual generation of up-calls may be controlled by the
\texttt{enableUpcalls} and the \texttt{disableUpcalls} method of the
\texttt{ReceivePort} object and are initially disabled. If an up-call
handler was provided when the receive port was created but the up-call
generation is currently disabled, an incoming message may be extracted
using the regular synchronous receive methods. If the up-call
generation is currently enabled, up-calls get the priority over
synchronous receive calls. As a result, any synchronous receive call
will not be served until the up-call generation gets disabled and a new
incoming message is detected. Nevertheless, programs should probably
avoid using synchronous receive calls while up-call generation is
enabled, as a general rule.

%
\subsubsection{Clean-up}
\label{sec:clean-up-2}

The send ports and receive ports should be disposed of using their
\texttt{free} method when not used anymore. It is up to the
application to ensure that no more messages are to be expected on a
given receive port and that it is safe to close it.

Once all the send ports and receive ports have been disposed of, the
\Ibis instance itself may be discarded by calling its \texttt{free}
method. Actually, these clean-up calls are not strictly mandatory but
are considered good practice and could be used in the future to tell
an unexpected connection lost from a regular end-of-session connection
close.


\subsection{Example}
\label{sec:example}

\begin{figure}[htbp]
  \centering
\begin{Miniverb}
import ibis.ipl.*;

public final class Mini {
  public static void main(String [] args) {
    try {
      Ibis ibis         = Ibis.createIbis(null);
      PortType t        = ibis.createPortType("example_port", null);
      Registry registry = ibis.registry();
      IbisIdentifier id = ibis.identifier();

      if (id.equals(registry.elect("example_master", id))) {
        SendPort sport = t.createSendPort();
        ReceivePortIdentifier ident = null;

        do {
          ident = registry.lookupReceivePort("example_slave_port");
          if (ident == null) {
            Thread.sleep(500);
          }
        } while (ident == null);
        sport.connect(ident);

        WriteMessage out = sport.newMessage();
        out.writeByte((byte)1);
        out.send();
        out.finish();

        sport.free();
      } else {
        ReceivePort rport = t.createReceivePort("example_slave_port");
        rport.enableConnections();

        ReadMessage in = rport.receive();
        byte b = in.readByte();
        System.err.println("b = "+b);
        in.finish();

        rport.free();
      }

      ibis.end();
    } catch (Exception e) {
      throw new Error(e);
    }
  }
}
\end{Miniverb}

  \caption{A Minimalistic Example}
  \label{fig:mini-example}
\end{figure}

Let's now have a look at a minimalistic but complete example. This
example assume a 2-processes session where one process (the
\emph{master}) sends a single byte to the other process (the
\emph{slave}). The full listing is presented on
figure~\ref{fig:mini-example}. The first line declares the imports
required to use Ibis.
\begin{Miniverb}
import ibis.ipl.*;
\end{Miniverb}

Then, we initialize \Ibis and our port type. In this example, we do
not specify any run-time properties for the port type, so all the
properties will be read from the configuration file.
\begin{Miniverb}
      Ibis ibis  = Ibis.createIbis(null);
      PortType t = ibis.createPortType("example_port", null);
\end{Miniverb}

After that we retrieve the \emph{registry} and the local \Ibis
instance \emph{identifier} and we use the registry to elect the
\emph{master} process (and hence the \emph{slave} process too).
\begin{Miniverb}
      Registry registry = ibis.registry();
      IbisIdentifier id = ibis.identifier();

      if (id.equals(registry.elect("example_master", id))) {
        . . .  // Master part
      } else {
        . . .  // Slave  part
      }
\end{Miniverb}
The registry symbolizes an access to the session \emph{name server}.
It can be used to perform various interprocess operations, like
organizing elections as above.

The first parameter of the \texttt{elect} method is the name of the
election, used to identify this election, and the second argument is
an \texttt{Object} which will be returned to all processes
participating to the election if the current process wins the
election.

An election is not a barrier and participating processes do not wait
for each other. This means that several elections may be conducted
simultaneously as long as each election has a unique identifier. An
election identifier may not subsequently be reused during the same
session.

The master side creates a send port. It then looks up the slave
receive port's identifier, again using the registry. Once the receive
port identifier has been found, the master connects its send port to
the slave's receive port.
\begin{Miniverb}
        SendPort sport = t.createSendPort();
        ReceivePortIdentifier ident = null;

        do {
          ident = registry.lookupReceivePort("example_slave_port");
          if (ident == null) {
            Thread.sleep(500);
          }
        } while (ident == null);
        sport.connect(ident);
\end{Miniverb}

Meanwhile, the slave creates a receive port and enable incoming
connections.
\begin{Miniverb}
        ReceivePort rport = t.createReceivePort("example_slave_port");
        rport.enableConnections();
\end{Miniverb}

After the master send port has been successfully connected to the
slave receive port, the actual communication can start. On the master
side, the process first creates a new outgoing message. Using the
\texttt{WriteMessage} object it just obtained, it appends a byte of
data to this message. Just in case the underlying layer was waiting
for sending the message, the master process calls \texttt{send}. And
finally, it waits for the transfer completion by calling
\texttt{finish}.
\begin{Miniverb}
        WriteMessage out = sport.newMessage();
        out.writeByte((byte)1);
        out.send();
        out.finish();
\end{Miniverb}

On the other side, the slave process performs a synchronous
\texttt{receive} call in order to wait for the master's message. Upon
message arrival, the \texttt{receive} method returns a
\texttt{ReadMessage} object which the slave use to extract the data
byte from the message and subsequently complete the reception by a
\texttt{finish} call.
\begin{Miniverb}
        ReceivePort rport = t.createReceivePort("example_slave_port");
        rport.enableConnections();
\end{Miniverb}

To conclude, the master closes its send port and the slave does the
same with its receive port.
\begin{Miniverb}
    . . .
        // Master part
        sport.free();
    . . .
        // Slave  part
        rport.free();
    . . .
\end{Miniverb}

Finally, the session ends with the termination of the \Ibis instance.
\begin{Miniverb}
      ibis.end();
\end{Miniverb}

%______________________________________________________________________

%%%%
\chapter{Programmer's Guide}
\label{cha:programmers-guide}

This chapter presents some technical information about the \NetIbis
internals. It explains the current organization of the existing code
and gives some details about how to extend the functionality of
\NetIbis through writing new drivers.

%%%
\section{Overview}
\label{sec:overview}

We start with a general overview of \NetIbis from the internal point
of view. This tour of the internals is completed with a set of
programming guidelines that should help keeping new added code
consistent with existing code.

%%
\subsection{Main Architecture}
\label{sec:main-architecture}

\NetIbis is basically organized in a two-layer fashion. The upper
layer implements the \Ibis Programming Layer interface (IPL) in a
generic manner independent from underlying networking hardware
details. The lower layer is responsible for dealing with these
details. It is made of a set of \emph{driver boxes} that may be used
together to build a driver tree fitting both the networking resource
characteristics and the actual specific application needs.

%%
\subsection{Programming guidelines}
\label{sec:progr-guid}

Before going further into \NetIbis internals' details, a few general
guidelines should be followed when adding code into \NetIbis. These
guidelines are presented below, without any specific order.

% - name all the threads

\paragraph{Name all the threads.}
\label{sec:name-all-threads}
\NetIbis uses a \textbf{lot} of threads for many kinds of jobs.
Uniquely naming new threads is therefore important to simplify
debugging sessions and to detect blocked threads at the end of the
session.

% - avoid non-interruptible locks

\paragraph{Avoid non-interruptible locks.}
\label{sec:avoid-non-interr}

\NetIbis has been designed to provide a minimum level of network-fault
tolerance. That especially means that no part of the \NetIbis code
should be irrecoverably locked when a internal thread is waiting for
some event on a given connection and that connection suddenly needs to
be closed due to a network failure. The various locks provided by the
\NetIbis distribution (see~\ref{sec:locks}) should be safe in that
respect. On the contrary, the \textsc{Java}'s \texttt{synchronized}
keyword is not safe in that respect as it is not interruptible and as
result should be used carefully. Typically, this keyword should only
be used in cases where the synchronization does not depend from
successfully sending or receiving something on the network.

% - propagate close and interrupt events

\paragraph{Propagate close and interrupt events.}
\label{sec:prop-close-interr}

This guideline is directly related to the previous one. It requires
\texttt{InterruptedException} interrupts and ``closed connection''
notifications to be propagated using the
\texttt{NetIbisInterruptedException} and the
\texttt{NetIbisClosedException} respectively. Proper propagation of
these exception is critical to the network-fault tolerance support in
\NetIbis.

% - beware of multiple caching levels

\paragraph{Beware of multiple caching levels.}
\label{sec:beware-mult-cach}

The highly modular organization of \NetIbis may lead to unexpected
effects if all modules get equipped with various kind of memory
caching mechanisms without taking into account other existing memory
block caching pools. As a rule of thumb, caching should be done using
class-level pools (that is \texttt{static} entities in the
\textsc{Java} meaning) instead of instance-levels pools. 

%%%
\section{User Programming Interface}
\label{sec:user-progr-interf}

This section describe the upper layer of the \NetIbis architecture
which implements the user visible programming interface and the
various objects involved in building this layer.

%%
\subsection{Connections}
\label{sec:connections-1}

While not directly visible from and manipulated by the user, the
\emph{connection} is the fundamental concept in \NetIbis networking.
It symbolizes an isolated and unidirectional virtual networking link
going from a source to a destination node. A \NetIbis connection is
FIFO-ordered and completely isolated from other connections, unless
the underlying networking environment used by this connection prevents
it.

\NetIbis connections actually are made of two network links. The
\emph{application} link is a unidirectional link and uses the
networking software/hardware selected by the user. It is exclusively
used by the application. The \emph{service} link is a bi-directional
link made of a pair of streams\footnote{TCP streams in the current
  \NetIbis implementation.} and may be used by the \NetIbis internals
to exchanges data between the source and destination nodes of the
connection, for connection setup information transfer and
synchronization, and for features involving in-session dynamic
negotiation.

In order to prevent dead-locks and to avoid the needs of complex
synchronizations, the service link input and output streams of a
connection are not accessed directly. Instead they are indirectly
accessed through multiplexed virtual sub-streams that can be
dynamically allocated at any time during the session. Each virtual
sub-stream is made of a \Java object stream transparently multiplexed
with other existing virtual sub-streams on the actual input or output
service link stream.

%%
\subsection{SendPort Objects}
\label{sec:sendport-objects}

The \texttt{NetSendPort} class implements the \texttt{SendPort} and
\texttt{WriteMessage} interfaces of the IPL. The reason for not
providing a separate class in order to implement the
\texttt{WriteMessage} interface is that both interfaces actually
requires access to almost the same set of class members. As a result,
separating the \texttt{NetSendPort} class into two classes would imply
either to copy a lot of data from the \texttt{SendPort} instance to
each new instance of the \texttt{WriteMessage} class or to give each
\texttt{WriteMessage} instance an access to the \texttt{SendPort} data
members. Both solutions are therefore not really interesting.

The \texttt{NetSendPort} object also implements two internal \NetIbis
interfaces.  The first one is the \texttt{NetPort} interface and
the second one is the \texttt{NetEventQueueConsumer} interface.

The \texttt{NetPort}'s purpose is to provide a unified way to
reference both \texttt{NetSendPort} and \texttt{NetReceivePort}
objects in contexts where the actual nature of a given port does not
matter. This \texttt{NetPort} interface currently provide the
\texttt{getPortType} method.

The \texttt{NetEventQueueConsumer} interface implementation is the means
through which a \NetIbis class enables itself to subscribe to a
\texttt{NetEventQueue} instance using a \texttt{NetEventQueueListener}
object. In the current \NetIbis send port implementation, event queues
are used to asynchronously inform the send port about connection close
events.

%%
\subsection{ReceivePort Objects}
\label{sec:receiveport-objects}

Like the \texttt{NetSendPort} class, the \texttt{NetReceivePort}
implements the \texttt{NetPort} and the \texttt{NetEventQueueConsumer}
interfaces. Symmetrically with respect to the \texttt{NetSendPort}
class, it also implements the \texttt{ReceivePort} and
\texttt{ReadMessage}. On top of that, the \texttt{NetReceivePort}
class implements a fifth interface called \texttt{NetInputUpcall}.

The \texttt{NetInputUpcall} interface provide an entry point for each
lower layer drivers to trigger incoming message up-calls in the
\texttt{NetReceivePort}. The \texttt{NetInputUpcall} interface export
a single method \texttt{inputUpcall}. In the \texttt{NetReceivePort},
the implementation of this method is, among other thing, responsible
of triggering the user defined incoming message up-call.

%%%
\section{Driver Programming Interface}
\label{sec:driv-progr-interf}

The lower layer of \NetIbis is responsible for interfacing the generic
IPL implementation provided by the upper \NetIbis layer with the
networking resources. It is built out of boxes called \emph{drivers}.
Each driver box may itself provide \emph{input} boxes and
\emph{output} boxes. Input and output boxes are used to actually carry
network connections (see Section~\ref{sec:connections-1}).

The programming interfaces of these three different sets of boxes
(driver, input and output) are uniform in each set. The consequence of
this characteristic is that all drivers are used the same way and are
therefore transparently exchangeable (unless their semantics prevents
this\footnote{For instance, a driver managing a physical network will
  always be in the lowest position}). The same is true for input and
output boxes.

Each input (resp. output box) may control one or more input (resp.
output) boxes and optionally one or more output (resp. input) boxes
from possibly different drivers. An example of output object managing
several output is a multi-cast output. An example of an output object
managing both an output and an input is a reliability driver's output
that would use the output for its main data connection and the input
for the reverse acknowledgment connection. Finally a driver could do
the other way round and mix several output or input's connections into
one single multiplexed output or input.

%%
\subsection{Driver Objects}
\label{sec:driver-objects}

Each driver should provide a \texttt{Driver} class extending the
\texttt{NetDriver} abstract class and located in a sub-directory of
the \NetIbis \texttt{"net"} directory. This \texttt{Driver} class will
be instantiated at most once per \NetIbis session.\footnote{Actually,
  each driver is instantiated at most once by a \texttt{NetIbis} \emph{instance}
  so, theoretically, a process running more than one \texttt{NetIbis}
  instance could have more than one instance of the same driver but
  running multiple \texttt{NetIbis} instances in the same process is
  not officially supported.}

The \texttt{Driver} class is used to manage the state of each
networking object and related networking resource controlled by this
driver at any time during the session. This means that an instance of a
\texttt{Driver} class has a global view of the entities involved in
networking with this driver.

Each \texttt{NetDriver} subclass should define the following three
methods that are declared \texttt{abstract} in \texttt{NetDriver}.
\begin{itemize}

\item The method \texttt{getName} should return the name of the
  driver. The driver's name should match the driver's directory name.

\item The method \texttt{newInput} is called in order to create a new
  driver-specific sub-class instance of the \texttt{NetInput} class.

\item Similarly, the method \texttt{newOutput} is called when the
  creation of a new driver-specific sub-class instance of the
  \texttt{NetOutput} class.

\end{itemize}
The \texttt{newInput} and \texttt{newOutput} methods both take a
port-type parameter and a context-string parameter in order to
specialize the communication object creation according to the port
settings and to the position of the new I/O object in the port's driver
hierarchy.

%%
\subsection{Input and Output}
\label{sec:input-output}

The input and output objects are responsible for managing the network
connections. Each object may control one or more connection. Each
connection may go through one or more objects before reaching the
network.

%
\subsubsection{Common Characteristics}
\label{sec:comm-char}
An I/O object is providing an \emph{added value} (possibly empty)
applied to the data passing through the connections it manages. The
series of output objects encountered by an outgoing network connection
should match the reversed series of input objects encountered on the
corresponding incoming connection on the other side.

%
\subsubsection{Output Objects}
\label{sec:output-objects}


\paragraph{Single Added Value Concept}
\label{sec:single-added-value}

An output object is supposed to take care about one or more outgoing
connections. The actual way these connections are taken care of is up
to the implementation. Here are some examples of scenarios:
\begin{itemize}

\item Put the data carried through each connection onto the network.
  Here, the considered output object is acting as a network driver.

\item Forward all the connections to a single lower-level output
  object. Here, the considered output object is acting as filter on
  the data.

\item Forward the connections as one or more sub-sets to as many
  sub-output objects. Here, the considered output object is acting as
  multi-cast splitter.

\item Forward all the connections onto one single sub-connection
  passed to a single sub-output object. Here, the considered output
  object is acting as a connection multiplexer.

\item Provide reliability by using both outgoing sub-connections for
  data and incoming sub-connections for acknowledgments.

\end{itemize}

It is advised to limit the \emph{action} performed by an output object
to a single added-value to keep each output object as simple as
possible. Moreover, taking care of writing an output object without
any assumption on the internals of sub-connection (both outgoing and
incoming ones) will ensure that this object can be used anywhere in
the communication stack\footnote{Of course, there are particular
  cases, as for example the network drivers which always are at the
  lowest place in the communication stack.}.


\paragraph{Data Types}
\label{sec:data-types}

Each output object is supposed to be able to process all the Java
types, that is the native types, the arrays and the objects. The
\texttt{NetOutput} abstract class, which is the base class of each
output object, provide a set of default methods that will convert
every type of data to series of single bytes so that basic output
objects only have to provide a method for sending bytes.

Of course, more advanced output objects may override the sending
methods for other types, either for efficiency or because the added
value they provide implies it. The \texttt{NetBufferedOutput} abstract
class, for instance, provides a default behavior for output objects
that work on byte arrays. It is more efficient than the
\texttt{NetOutput} base class for this specific use because it
overrides the methods for sending aggregates of bytes with versions
that process these aggregates directly instead of splitting those
aggregates into single bytes.

It should be noted that overriding the methods for writing various
types of data should be done carefully to avoid cycles. For instance,
a method converting an array of bytes to single bytes which in turn
calls a method that aggregates multiples bytes into arrays of bytes
and so on would obviously not work.


\paragraph{Communication}
\label{sec:communication}

The communication unit of \Ibis and \NetIbis is the \emph{message}.
The message unit concept is also found in the interface of the
\texttt{NetOutput} classes. An \texttt{NetOutput} object instance may
only send one single message at a time. An other way to put it is that
a message is considered atomic from the point of view of the user of a
\texttt{NetOutput} object instance. When a \texttt{NetOutput} instance
manages more than one connection, each message it process is supposed
to be traveling on all the connections at the same time.

A message is initiated with the \texttt{initSend} method. Then, the
actual message contents is constructed incrementally by using the
\texttt{write*} methods. The \texttt{NetOutput} instance is not
required to bufferize the pieces of message until the message
completion and may freely decide whether to immediately send a piece
of data or at a later point in time. The message completion is reached
when the \texttt{finish} method is called.

It should be stressed that the atomic characteristic of a message
managed by a given \texttt{NetOutput} instance is only required to be
true from the point of view of code \emph{using} this instance. The
instance is not required to preserve that atomicity \emph{internally}
and the message may well be sent as a series of sub-messages
interleaved with sub-messages from other messages (coming from other
\texttt{NetOutput} instances) if needed, when several output objects
are multiplexed, for instance.

\paragraph{Initialization and Clean-Up}
\label{sec:init-clean-up}

Initializing a \texttt{NetOutput} object for communication involves
two steps. The first one is the construction of the object itself, the
second one is the initialization of the connection(s) managed by the
object.

A \texttt{NetOutput} object should be constructed through a call to
the \texttt{newOutput} method of its related \texttt{Driver} class. It
should be avoided to construct the object directly. Failing to do so
would bypass its \texttt{Driver} class and therefore would probably
prevent the drivers that rely on this feature such as multiplexing
drivers to work properly.

The constructor of the \texttt{NetOutput} subclasses should require at
least a \texttt{NetPortType} argument providing access to the settings
of the enclosing communication port, a \texttt{NetDriver} argument
referencing the output's related driver instance and a \texttt{String}
argument containing the output's creation context in order to ensure
subsequent settings queries to access the right settings. The
constructor must also invocate its superclass constructor with these
arguments.

Each new connection creation involves the recursive calls of the
\texttt{setupConnection} method of each \texttt{NetOutput} object it
goes through. By default, this method takes a single argument, a
\texttt{NetConnection} reference to the connection being created.

If a \texttt{NetOutput} object needs to exchange some \emph{meta} data
with its peer \texttt{NetInput} object on a given connection, either
during the connection setup or any time later while the connection is
still active, it has the option to allocate a stream on the
connection's \emph{service link}'. See Section~\ref{sec:connections-1}
for a description of the service link. The service link itself can be
accessed through the \texttt{getServiceLink} method of the
\texttt{connection} object. This method returns a
\texttt{NetServiceLink} object. The stream itself may be allocated
using the \texttt{getOutputSubStream} method or the
\texttt{getInputSubStream} method of the \texttt{NetServiceLink}
object for allocating a out-going or a in-coming stream respectively.
Both methods expect a \texttt{name} argument which will should
identically be specified by the peer \texttt{NetInput} object for
allocating the other end of the stream. The name is automatically
prefixed with the prefix of the \texttt{NetOutput} object
(\texttt{NetInput} on the other side). As a result, a stream name only
needs to be unique inside a given \texttt{NetOutput/Input} object for
a given connexion and a given stream direction.

If the service link of a connection happens to be lost either because
the peer connection side has been closed intentionally or not, the
connection is considered to have been closed. It should therefore be
stressed that there is no difference made between a broken connection
and a regularly closed connection.

%
\subsubsection{Input Objects}
\label{sec:input-objects}

The \texttt{NetInput} objects work in a way very similar to that of
the \texttt{NetOutput} objects as far as connection setup and data
carrying are concerned. Hence, we will presents a few relevant details
here.


\paragraph{Initialization}
\label{sec:initialization-1}

In contrast to many existing communication tools, \NetIbis does not
require the \texttt{NetInput} objects to take care of the listening
for new incoming connections. This step is controlled by the \NetIbis
core code as part of the management of service links. As a result,
when a new incoming connection is setup through a series of recursive
\texttt{SetupConnection} calls on the \texttt{NetInput} objects used
by this connection, each \texttt{NetInput} object may assume that
there is a peer \texttt{NetOutput} object on the other side which is
ready to connect. Pairs of peered \texttt{NetInput}/\texttt{NetOutput}
objects may allocate and use one or more sub-streams of the service
link, in order to exchange information about the connection being setup.


\paragraph{Asymmetrical Behavior}
\label{sec:asymm-behav}

The main difference between a \texttt{NetInput} object and a
\texttt{NetOutput} object is that a \texttt{NetOutput} object is
supposed to multi-cast each message to all the connections it manages.
On the other hand, a \texttt{NetInput} is expected to serialize
incoming messages from multiple connections. An other way to put it is
that in a \texttt{NetOutput} object, all the connections are either
active or inactive at the same time, while at most one connection is
allowed to be active at any time in a \texttt{NetInput} object.
This asymmetrical scheme is actually quite natural as it would not
make sense to mix data coming from multiple connections.


\paragraph{Communication}
\label{sec:communication-2}

As for the \texttt{NetOutput} objects, messages are also considered
atomic from the point at the interface level of the \texttt{NetInput}
objects. An atomic receive sequence begins with a \emph{successful}
call to the \texttt{poll} method of the \texttt{NetInput} object and
ends with a call to the \texttt{finish} method of that same object.

It should be stressed here that the atomic sequence \emph{only} begins
in the case that the \texttt{poll} call is successful. Otherwise,
\texttt{poll} has to be repetitively called again until it succeeds.
As a convenience, the \texttt{poll} method takes a boolean as its
single argument which may be set to \texttt{true} in order to request
a \emph{blocking} poll. In that case, the \texttt{poll} method only
returns on success or if the connection has been closed or broken.

It should equally be stressed that if the \texttt{poll} method
succeeds, the corresponding message \emph{must} be extracted
completely before attempting subsequent new poll requests. Calling
\texttt{poll} before the message has fully been extracted or just
after a successful poll will result in unspecified behavior and most
likely in communication stream corruption.

Nevertheless, as for the case of the \texttt{NetOutput} objects, a
pair of \texttt{poll}/\texttt{finish} method calls on a given
\texttt{NetInput} object may be mapped on multiple pairs of
\texttt{poll}/\texttt{finish} calls to the \texttt{NetInput} object
below and these smaller atomic sequences may be interleaved with other
atomic sequences.

%%%
\section{Drivers}
\label{sec:drivers}

This section presents the currently available drivers in \NetIbis and
gives some details about their implementation. The first subsection is
dedicated to the network drivers also called physical drivers. The
other subsections deal with the various kind of virtual drivers.

%%
\subsection{Network Drivers}
\label{sec:network-drivers}

The network drivers are responsible for managing the configuration networking
resources used during the computing session. They are therefore always
put at the deepest level in the driver tree.

%
\subsubsection{Generic Classes}
\label{sec:generic-classes}

Most networking software/hardware combinations have in common to only
be able to carry blocks of bytes. As a result, most \NetIbis network
driver implementations would have to provide very similar byte block
building logics to prepare pieces of data for network travel (remember
that by default, the \NetIbis generic \texttt{NetOutput} class reduce
every piece of data to a flow of single bytes).

In order to avoid code duplication among most network drivers,
\NetIbis provides a pair of
\texttt{NetBufferedInput}/\texttt{NetBufferedOutput} abstract classes.
These classes override the necessary \texttt{NetInput} and
\texttt{NetOutput} methods in order to reduce the flow of data pieces
to a flow of bytes blocks. These bytes blocks are called buffers,
hence the name of the classes. This pair of classes may also be used
in building virtual drivers. It is not dedicated to network drivers.
Nevertheless, it is mentioned here because most network drivers rely
on it.


\paragraph{NetBufferedOutput}
\label{sec:netbufferedoutput}

The \texttt{NetBufferedOutput} abstract class provide the default
behavior for sending bytes buffers over the network or to a lower
\texttt{NetOutput} object. A non-abstract \texttt{NetBufferedOutput}
sub-class must implement at least the \texttt{setupConnection}, the
\texttt{sendByteBuffer} and the \texttt{close} methods.

The \texttt{sendByteBuffer} method takes a reference to a
\texttt{NetSendBuffer} or sub-class object. In the current \NetIbis
implementation, the \texttt{NetSendBuffer} class is only a wrapper
over an array, which means that it's purpose is not to provide
encapsulation but instead to attach some meta-data to the byte array
used for buffering. As a result, most fields of the
\texttt{NetSendBuffer} class are supposed to be accessed directly.
Until this interface is revised, sub-classes of the
\texttt{NetSendBuffer} class should be carefully designed to use the
public fields of \texttt{NetSendBuffer} in a way consistent with the
default \texttt{NetSendBuffer} implementation.

The main public fields of a \texttt{NetSendBuffer} class are the
\texttt{data} field, which is the byte array used for buffering, the
\texttt{base} field indicating the first byte to send in the buffer
and the \texttt{length} field the total length of data in the buffer.
Subclasses of the \texttt{NetBufferedOutput} class should set their
\texttt{mtu} --- maximum transfer unit --- field to the preferred (and
maximum) size of buffer it is able to send as a single piece. If the
managed network may carry buffers of any size, the \texttt{mtu} field
should be set to 0, which means no mtu.


\paragraph{NetBufferedInput}
\label{sec:netbufferedinput}

The \texttt{NetBufferedInput} abstract class provide the same kind of
service than the \texttt{NetBufferedOutput} class but for the
receiving side. A non-abstract \texttt{NetBufferedInput} sub-class
must implement at least one of the two \texttt{receiveByteBuffer}
methods as well as the following methods.
\begin{itemize}
\item \texttt{SetupConnection}.
\item \texttt{doPoll}.
\item \texttt{doFinish}.
\item \texttt{doClose}.
\item \texttt{doFree}.
\end{itemize}
If the \texttt{initReceive} method is overridden, the super-class
\texttt{initReceive} method should be called somewhere in the
overriding \texttt{initReceive} body.

Two versions of the \texttt{receiveByteBuffer} method are available.
One version takes a \texttt{NetReceiveBuffer} object that wraps a byte
array that should be filled by the method. The other version takes an
\texttt{expectedLength} integer and should return a
\texttt{NetReceiveBuffer} containing a piece of data read from the
network. The \texttt{expectedLength} integer is only a hint for the
network drivers that do not send meta-data indicating the length of
each piece of data as part of the network data flow. 

If only one of the two versions of this method is overridden, a
default implementation of the other one is provided on top of the one
that is overridden. If none of the two methods are overridden, an
\texttt{error} is thrown at run time as this would create a cycle
between the two default implementations. It is not possible to detect
this error at compile-time.

%
\subsubsection{Default Driver}
\label{sec:default-driver}

If no driver is specified for the driver tree, the \texttt{default}
driver is automatically selected in a single node tree. This driver is
not designed to be efficient. Each network connection opened with this
driver use its own service link to carry both any internal service
stream allocated for this connection (as usual) as well as the
application data stream corresponding to this connection.

Because the service link is always open (over TCP in the current
\NetIbis implementation) for each network connection, the
\emph{default} driver is guaranteed to work in any situation where
\NetIbis as a whole has a chance to work. It should also be noted that
this driver uses less networking resources than the other ones because
of the multiplexing of the data stream with the service streams on
each connection. The cost of using less network resources is of course
that the data stream bandwidth is shared with service streams and that
the data stream latency includes the transfer of the multiplexing
meta-data.

%
\subsubsection{TCP Drivers}
\label{sec:tcp-drivers}

The current \NetIbis sources provide three different implementations
of a TCP driver: the regular one (simply called TCP driver), the
\emph{plain} TCP driver and the \emph{block} TCP driver.

\paragraph{TCP Driver}
\label{sec:tcp-driver}

This driver is an example of a network driver providing communication
routines for each kind of \textsc{Java} data type. It is built using a
pair of \texttt{DataOutputStream}/\texttt{DataInputStream} on top of
the socket stream for each connection. It is more designed as a
starting point for writing other similar network drivers than as a
driver for production use.

\paragraph{TCP `plain' Driver}
\label{sec:tcp-plain-driver}

The \emph{plain} version of the TCP driver is also an example of a
network driver but this time a minimalistic one. It relies on the
\NetIbis internal automatic reduction of data types to a stream of
single bytes. It is therefore mostly made of a method sending single
bytes and a method receiving single bytes. Obviously, it is not
dedicated to efficiency but, like the regular TCP driver, to provide a
starting point for quick prototyping of new network drivers. 

\paragraph{TCP `block' Driver}
\label{sec:tcp-block-driver}

The \emph{block} version of the TCP driver is both a driver for real
applications and an example of a driver built on top of the
\texttt{NetBufferedOutput}/\texttt{NetBufferedInput} pair of abstract
classes. With this driver, each piece of data is accumulated as a
block of bytes into a fixed-size buffer. This buffer is sent as soon
as it gets completely filled, or upon a \texttt{finish} call at the
end of a message construction. In the current implementation of this
driver, the size of the buffers is set to 32~KB.


%
\subsubsection{GM Driver}
\label{sec:gm-driver}

Like the \emph{block} version of the TCP driver, the GM driver is also
based on the \texttt{NetBufferedOutput}/\texttt{NetBufferedInput} pair
of abstract classes. The size of the buffers is set to 2~MB.
GM is able to send packets up to 4~GB, so the driver should
theoretically not enforce a 2~MB boundary on the size of the buffers
but we want to be able to maintain a cache of allocated buffers to
avoid multiple expensive allocation and free of buffers. 

It should be noted that, in contrast with the TCP \emph{block} driver,
the GM driver has a zero copy capability. As a result, 2~MB buffers
will only be allocated in the situations where a copy cannot be
avoided (when an object has to be serialized, for instance). If a user
application block is eligible for zero-copy transmission but is larger
than 2~MB, it is automatically sent using multiple shorter zero-copy
transmissions.

Internally, the GM driver may send packets using two different
protocols. Small packets up to 4~KB are sent using an \emph{eager}
protocol. Larger packets are sent using a \emph{rendez-vous} protocol.

The \emph{rendez-vous} protocol transmission is done in three phases.
First, a \emph{request} packet is sent by the sender node to the
receiver node. Once the receiver node is ready to receive the piece of
data, it answers the request from the sender with an
\emph{acknowledgment} packet. Upon receiving the acknowledgment
packet, the sender node sends the actual piece of data as a raw
\emph{data} packet to the receiver node.

The \emph{eager} protocol transmission only uses the first step of the
\emph{rendez-vous} protocol. As a counter part, the piece of data is
copied directly in the \emph{request} packet.

The GM API does not provide any multiplexing mechanism with the
exception of the communication ports which should be reserved to
process level multiplexing. On the other hand, the GM API provides
means to selectively receive high or low priority packets which
indirectly results in disposing of two independent flows of packets.
As a result, in order to avoid receiving a \emph{request} packet or a
\emph{acknowledgment} packet when a \emph{data} packet is expected in
the \emph{rendez-vous} protocol, the \emph{request} and
\emph{acknowledgment} packets are sent using the
\texttt{GM\_HIGH\_PRIORITY} attribute while the \emph{data} packets are
sent using the \texttt{GM\_LOW\_PRIORITY} attribute.

The GM driver of \NetIbis has been designed to rely completely on the
\textsc{Java} virtual machine as far as the synchronization and
multi-threading mechanisms are concerned. This design orientation,
which might induce some performance penalty was followed in order to
avoid the direct use of these often heavily unportable constructions.

%
\subsubsection{Pipe Loop-back Driver}
\label{sec:pipe-loopback-driver}

The \emph{Pipe} driver of \NetIbis is dedicated to in-process
loop-back communication. It is built on the \texttt{PipedInputStream}
and \texttt{PipedOutputStream} objects of the \textsc{Java} library.
This driver is not supposed to be used to build connections across
multiple processes, even on the same physical node. It is only
dedicated to communication between threads of the same process in
multi-threaded applications. According to the specifications of the
\texttt{PipedInputStream} and \texttt{PipedOutputStream} classes in
the \textsc{Java} library, it is not safe to send and receive data
on the same piped stream within the same thread as the internal buffer
of the piped stream could get filled and result in a dead-lock. 

With the exception of the use restrictions just mentioned, the
architecture of the \emph{pipe} driver is basically the same as the
architecture of the \emph{block} TCP driver. There is only a slight
difference in the \texttt{setupConnection} method implementation.
Indeed, the piped stream is not socket based. In order to connect a
\texttt{PipedOutputStream} instance to a \texttt{PipedInputStream}
instance, you need to apply the \texttt{connect} method of the
\texttt{PipedInputStream} instance to the instance of the
\texttt{PipedOutputStream}. The problem here is that it is not
possible to send the \texttt{PipedOutputStream} object through the
service link as it is done with the \texttt{address} and \texttt{port
  number} in the case of the TCP driver. So, instead the \emph{pipe}
driver uses the \NetIbis \emph{bank} facility (see~\ref{sec:bank}).

As a result, the thread responsible of initializing the sending side
of the pipe stream simply store its \texttt{PipedOutputStream}
instance into the \NetIbis bank and send the corresponding key to the
receiving side through the service link. Then, the receiving side
retrieve the \texttt{PipedOutputStream} instance from the bank using
the key and connect it to its \texttt{PipedInputStream} instance.

%
\subsubsection{UDP Driver}
\label{sec:udp-driver}
The \NetIbis UDP driver is yet another straightforward buffered driver
very similar to the TCP \emph{block} version. The only significant
difference is that the UDP driver does not provide reliability (which
is achieved in a generic manner using the \texttt{reliability} virtual
driver, see Section~\ref{sec:reliability-driver}), so packets may be
lost, duplicated or received out-of-order. No assumption should
therefore be made about which packet is going to be extracted after a
successful \texttt{poll} call and it should be avoided to receive
multiple packets in a row after a poll operation as there is no reason
for the packets of the series to actually be related, and hence no
reason to process the series of packets atomically.

%%
\subsection{Multi-Connection Drivers}
\label{sec:multi-conn-driv}
All the network drivers that we have seen so far share the
characteristics of being single connection oriented. Each instance of
their \texttt{Input} or \texttt{Output} classes will only manage one
single connection at a time (though that connection might very well
multiplex multiple streams if a multiplexing virtual driver is used
above in the driver stack).

We are now going to present two multi-connection drivers, namely the
\emph{generic multi-cast} driver and the \emph{multi-protocol} driver.
Both drivers actually share some code and hence derive from a common
set of multi-connection-specific abstract classes.

%
\subsubsection{Abstract Classes}
\label{sec:abstract-classes}
In order to avoid code duplication among the \emph{generic multi-cast}
driver and the \emph{multi-protocol} driver, and possibly future
multi-connection drivers, \NetIbis provides the \texttt{NetSplitter}
and \texttt{NetPoller} abstract classes. These classes manage the
common behavior shared by these multi-connection drivers.

The \texttt{NetSplitter} class is responsible for managing multiple
connections on the sending side. It \emph{splits} its set of
connections into smaller sets mapped on multiple (and possibly
identical) sub-trees of drivers. Each message constructed using a
\texttt{NetSplitter} instance during the session is then forwarded to
each sub-tree through its corresponding \texttt{NetOutput} object
instance. 

Each sub-tree entry-point \texttt{NetOutput} object is internally
stored in an \texttt{HashMap} object. Classes extending the
\texttt{NetSplitter} class basically need to provided a special
implementation of its \texttt{setupConnection} method that must call
the \texttt{NetSplitter}'s version of the \texttt{setupConnection}
with the connection instance, the (possibly new) \texttt{HashMap} key
of the sub-tree \texttt{NetOutput} object on which the connection is
to be forwarded and a reference to the \texttt{NetOutput} object
itself as arguments. \texttt{NetSplitter} sub-classes must also
provide a implementation for the abstract \texttt{closeConnection}
method which is responsible for cleaning-up a given connection.

The algorithm used for forwarding messages to sub-trees is not
optimized in any way. The \texttt{NetSplitter} generic methods simply
retrieve an \texttt{Iterator} over the internal \texttt{HashMap}
values, that precisely happen to be the sub-tree \texttt{NetOutput}
entry-point objects. In particular, the set of sub-\texttt{NetOutput}
objects is iterated \emph{sequentially}.

The \texttt{NetPoller} class is responsible for managing multiple
connections on the receive side. Unlike the \texttt{NetSplitter}
class, the \texttt{NetPoller} class processes incoming messages in a
sequential way roughly following the order of arrival of messages (see
Section~\ref{sec:asymm-behav}). It also introduces a notion of
\emph{receive queue} which encapsulates a sub-\texttt{NetInput}.

Classes extending the \texttt{NetPoller} class need to follow the same
guidelines as for the \texttt{NetSplitter} class. Moreover, they need
to provide a \texttt{selectConnection} which is called by the
\texttt{NetPoller} internals when a connection is activated upon a
polling request success. This method can therefore be used to execute
any required connection initialization stuff.

The rule used for building the sub-sets of connections is to be
provided by the non-abstract sub-classes of the \texttt{NetSplitter}
and must match the rule used on the \texttt{NetPoller} side. This rule
simply correspond to sub-input/sub-output key generation function.

%
\subsubsection{Generic Multi-cast Driver}
\label{sec:gener-mult-driv}
The \emph{generic multi-cast} driver implementation is very
straightforward, as most of the mechanisms are handled by the generic
methods of the \texttt{NetSplitter} and \texttt{NetPoller} abstract
classes. The key used for mapping a connection onto sub-input/output
objects is simply the \texttt{Integer} identifier of the connection
itself. Consequently, a \texttt{GenSplitter} instance generate one
sub-\texttt{NetOutput} object instance (of a unique class) for each
new connection. The same is true for the \texttt{GenPoller} class with
regard to sub-\texttt{NetInput} objects.

%
\subsubsection{Multi-protocol Driver}
\label{sec:multi-prot-driv}
The purpose of the \emph{multi-protocol} driver is to allow a driver
tree to split into multiple \emph{different} sub-trees. The rule for
sub-tree generation is provided by the means of plug-ins. The role of
such a plug-in is to generate \emph{sub-context strings} (see
Section~\ref{subcontext}, page~\pageref{subcontext}) according to some
given \emph{pattern}. These sub-context strings may then be mapped
onto driver sub-trees in the port configuration file. It should be
noted than new plug-in implementation should be carefully designed to
ensure that the instances of the plug-in on both side of a new
connection being setup take consistent decisions by selecting the
\emph{same} sub-context string. One way to do that is for example to
let only one of the two plug-in instances take the decision and send
the result to the other instance.

The implementation of the \emph{multi-protocol} driver is very close
to the implementation of the \emph{generic multi-cast} driver. The
main difference is that the key used to select sub-\texttt{NetInput}
or sub-\texttt{NetOutput} objects is the sub-context string in the
case of the \emph{multi-protocol} driver.

The current \NetIbis implementation provides three pattern generator
plug-ins for the \emph{multi-protocol} driver. We now briefly present
these plug-ins.

\paragraph{Random Plug-in}
\label{sec:random-plugin}
The \emph{Random} plug-in randomly selects one sub driver-tree out of
two for each new connection. This behavior is implemented by randomly
selecting a context string among \texttt{"a"} and \texttt{"b"}. Of
course, the purpose of this plug-in is more to demonstrate the
mechanism of plug-ins but with a good quality random number generator,
it might be used to balance a set of connections over two different
networks.

\paragraph{ABC Plug-in}
\label{sec:abc-plugin}
The \emph{ABC plug-in}'s purpose is also more dedicated to providing a
plug-in implementation example than a really useful piece of code. It
selects a context string according to some comparison of the IP
addresses of the two peer nodes. First, if the peer nodes are actually
one unique process, the plug-in selects the \texttt{"process"}
sub-context string. If not, the plug-in compares the bytes of both IP
addresses sequentially, outermost-to-innermost, until either all the
bytes successfully matched or the two bytes being compared do not
match. At each iteration, a counter (initially set to 0) is
incremented. Once the comparison loop is terminated, the plug-in
selects a context string according to the value of the counter. 
\begin{table}[htbp]
  \centering
  \begin{tabular}{p{4cm}p{6cm}}
    \hline
    \textbf{value}    & \textbf{string} \sep
    \texttt{0}        & \texttt{"internet"}\sep
    \texttt{1}        & \texttt{"net\_a"}\\
    \texttt{2}        & \texttt{"net\_b"}\\
    \texttt{3}        & \texttt{"net\_c"}\sep
    \texttt{4}        & \texttt{"node"}\sep
    Same process      & \texttt{"process"}\\
    \hline
  \end{tabular}
  \caption{ABC plug-in context strings}
  \label{tab:abc-strings}
\end{table}
The context strings for values from 1 to 3 are roughly inspired by
IP network \emph{classes}.

\paragraph{Cluster Plug-in}
\label{sec:cluster-plugin}
The \emph{cluster} plug-in is designed to allow \NetIbis to easily run
onto multiple clusters potentially interconnected with different
networks. When the cluster plug-in class is loaded in a \NetIbis
instance, the class-initialization code look-up a
\texttt{net\_multi\_cluster\_topology.txt} in order to determine to
which network(s) the current process has direct access. The
\texttt{net\_multi\_cluster\_topology.txt} file format is simply a
series of one or more network descriptions separated by a blank line.

A network description starts with a user-chosen network name
immediately followed by a colon on the first line and followed by a
series of fully qualified host names that have access to this network,
one per line. The series of machine names does not need to be complete
and the topology description file is not required to be unique among
all the nodes used in a given session.

Once a new connection is being setup, the sending side process
sequentially sends each network name it belongs to, in the order that
the networks where listed in its topology description file. For each
network name, the receive side process looks up his own table of
networks it belongs to. If the network name sent by the sender process
is found the in the receiver process table, it means that both
processes belongs to the same network and therefore the current
network name is selected as the sub-context string and there is no
attempt to find other matching networks in that case. On the contrary,
if the network name sent by the sender process is not found in the
receiver process, the sender process iterates to its next network and
so on. If no matching networks are found, no sub-context string is
selected which results in selecting the \emph{default} sub driver-tree
configured in the configuration file (the one that does not mention
any sub-context for the \emph{multi-protocol} driver).

%%
\subsection{Serialization Drivers}
\label{sec:seri-driv}
The serialization drivers are responsible for converting \textsc{Java}
objects, which are ``black boxes'' in nature, into pieces of data that
can be processed and carried onto the network, and used to reconstruct
the object on the other side. Just like the multi-connection drivers
above (see Section~\ref{sec:multi-conn-driv}), the source code common
to all the serialization drivers has been extracted as a pair of
abstract classes.

%
\subsubsection{Generic Classes}
\label{sec:generic-classes-1}
The \texttt{NetSerializedOutput} and \texttt{NetSerializedInput}
abstract classes provide the generic behavior common to the
serialization drivers. Typically, the \texttt{NetSerializedOutput}
class should be seen as some kind of wrapper that controls both the
interface and the resulting stream of a
\texttt{SerializationOutputStream}. It forces \textsc{Java} objects
(and more generally any \textsc{Java} data type) to enter the
\texttt{SerializationOutputStream}.

The \texttt{SerializationOutputStream} is an \textsc{Ibis} abstract
class extending the \texttt{ObjectOutputStream} class of the
\textsc{Java} library. The classes extending the
\texttt{NetSerializedOutput} class are required to provide a
\texttt{newSerializationOutputStream} that is responsible for
instantiating a subclass of the of the
\texttt{SerializationOutputStream}. The extension classes should also
provide some code to intercept the pieces of data resulting from the
processing performed by the \texttt{SerializationOutputStream}, in
order to push them deeper in the driver tree.

The \texttt{NetSerializedInput} does the opposite job on the other
side and extract \textsc{Java} objects (and more generally any
\textsc{Java} data type) from a \texttt{SerializedInputStream}
subclass instantiated and feed by a \texttt{NetSerializedInput}
subclass using pieces of data coming from the network.

%
\subsubsection{Ibis Serialization}
\label{sec:ibis-serialization}
The current \emph{Ibis serialization} driver uses the \textsc{Ibis}
serialization infrastructure located in the \texttt{/ibis/io}
directory in order to produce arrays of native types out of whatever
data is submitted for being transmitted. Technically, the
\texttt{SerializationOutputStream} used is an
\texttt{IbisSerializationOutputStream}. The
\texttt{IbisSerializationOutputStream} itself is initialized on top of
an \texttt{ArrayOutputStream} subclass provided by the \emph{Ibis
  serialization} driver of \NetIbis. The \texttt{ArrayOutputStream}
subclass allows the driver to catch the resulting arrays of the
\textsc{Ibis} serialization engine and forward them to the next
\texttt{NetOutput} object in the driver tree. The same organization is
used on the receive side.

%
\subsubsection{Sun Serialization}
\label{sec:sun-serialization}
The \emph{Sun serialization} driver on the other side is really
straightforward. The \texttt{SerializationOutputStream} used is a
\texttt{SunSerializationOutputStream} which in turn takes a
\texttt{OutputStream} subclass as argument at construction time. The
\texttt{OutputStream} subclass implementation is provided by the
driver and simply forward bytes to the next \texttt{NetOutput} object
in the driver tree.

%%
\subsection{Other Drivers}
\label{sec:other-drivers}
\NetIbis also provides a few other drivers that cannot really be put
in the previous categories. We now present these drivers.

%
\subsubsection{Identity Driver}
\label{sec:identity-driver}
The \emph{identity} driver is an example virtual driver. It can be put
anywhere in the driver tree (and even multiple time) with the
exception of the leaves --- the leaves are reserved to physical
drivers --- and it simply passes data untouched to the next
\texttt{NetOutput} object in the driver tree, or the previous
\texttt{NetInput} object on the receive side. Its purpose is basically
to test the modular driver structure of \NetIbis as well as port
configuration system.

%
\subsubsection{Reliability Driver}
\label{sec:reliability-driver}
TO-DO: Rutger

%
\subsubsection{Multiplexing Driver}
\label{sec:multiplexing-driver}
TO-DO: Rutger


%%%
\section{Other Features}
\label{sec:other-features}
The \NetIbis source tree also provides a few helper routines and
features that are not directly related to communication. These
features are presented in the following sections.

%%
\subsection{Debugging Support}
\label{sec:debugging-support}
\NetIbis features a logging mechanism in order to provide debugging
information about the code behavior. The logging mechanism is
implemented by the \texttt{NetLog} class. 

The \texttt{NetLog} class can be configured at compile time to select
the format of logged events, either \emph{human-readable} or not. Set
or clear the \texttt{human} boolean static field in the
\texttt{NetLog.java} file. The non-human-readable version provide much
more information on each event display and is formatted to be easy to
parse using so tools such as \textsc{Perl} scripts. 

The code of \texttt{NetLog} class can of coursed be tuned for both
modes to display the desired information. In the current
implementation, the \emph{non-human} version displays the following
information for each event.
\begin{enumerate}
\item A time-stamp.
\item The current thread name.
\item The "module name", which is a user-defined string that can be
  specified when the logging object is created.
\item The location of the event in the source code (class name, method, file
  and line number).
\item The memory state of the JVM.
\item The event message.
\end{enumerate}
The human-readable version displays the following information for each
event.
\begin{enumerate}
\item The class name (with its short package name).
\item The method name.
\item The event message.
\end{enumerate}

Moreover, each \texttt{NetLog} instance may be independently be turned
\texttt{on} or \texttt{off} at instantiation time. In particular, no
\texttt{NetLog} instance should be active on the communication
critical path in a \NetIbis session involving performance measurements
as logging an event is quite expensive, especially with the verbose
non-human-readable logging mode. On the contrary, this feature has
proved \emph{really} useful in many situations.

It should be noted that the \texttt{NetLog} class relies on JDK~1.4.x
features, namely the \texttt{getStackTrace} method of the
\texttt{Throwable} standard class which is not available and does not
seem to be implementable with earlier JDKs and the \emph{regular
  expression} support which is also unavailable in earlier JDKs (but
is only used cosmetically in \texttt{NetLog}). Nevertheless, the
logging commands may safely be commented out in the case the \NetIbis
needs to be run on earlier JDK versions.

The \texttt{NetLog} class may be used as-is for general purpose
logging. It is also the basis of a permanent logging infrastructure in
\NetIbis which is deployed in the \texttt{NetSendPort} and
\texttt{NetReceivePort} classes as well as in the \texttt{NetIO} class
and all its subclasses (hence, all the \texttt{NetInput} and
\texttt{NetOutput} classes, as \texttt{NetIO} is their common superclass).

This infrastructure may be used to log the \texttt{NetIbis}
communication internals' state while initializing, connecting and
communicating. The logging infrastructure activation control can be
managed from the \texttt{NetIbis} configuration file. Three levels are
provided.
\begin{enumerate}
\item Disp: dedicated to important logging information,
  \emph{activated} by default.
\item Trace: medium-level task-based logging information, \emph{not
    activated} by default. This level can especially be used to track
  messages as well as send port connecting to receive port, which is
  useful to find out information such as ``which particular message
  was lost'', ``which particular was just received before the node
  crashed'' or ``which send port failed to connect to which receive
  port''.
\item Log: low-level control-flow-based logging information, \emph{not
    activated} by activated. This level prints out information about
  events such as method entered and left and low level communication
  operations.
\end{enumerate}
Each level can be turned \texttt{on} or \texttt{off} by setting the
property of the same name to \texttt{true} or \texttt{false}
respectively. Each level can also be selectively activated on a driver
basis by prefixing the property name with a context path.

%%
\subsection{Stat Support}
\label{sec:stat-support}
The debugging support of \NetIbis described in the previous section is
completed by a simple infrastructure for gathering statistical
information about any kind of event or situation. This infrastructure
is provided by the \texttt{NetStat} abstract class.

The \texttt{NetStat} class provides a very basic framework for writing
classes that record information like the occurrence number of some
kind of event or the amount of some quantity and print a report at the
end of the session. The method exported by \texttt{NetStat} are
dedicated to the output of result values in the final report. These
methods are the \texttt{reportVal} and the \texttt{reportMap} methods.
The \texttt{reportVal} method simply prints a sentence presenting a
single value resulting from the statistical gathering throughout the
session. The \texttt{reportMap} method prints a series of sentences
presenting a series of results stored in a \texttt{HashMap} object
(unless the map is empty in which case nothing is printed).

The \texttt{report} method itself --- which is called at the end of
the session to report all the results gathered by the \texttt{NetStat}
subclass instance --- is abstract and must therefore be
provided by subclasses of \texttt{NetStat}. The calling of this method
at closing time is ensured by the \texttt{NetStat} constructor which
registers a wrapper of this method to the JVM using the
\texttt{addShutdownHook} method of the current instance of the
\texttt{Runtime} class.

Like the \texttt{NetLog} class, the \texttt{NetStat} class provides a
way to turn new instances \emph{on} or \emph{off} at construction
time. It also allows the programmer to specify a \texttt{moduleName}
argument to identify the instance location and purpose to the user.

The \texttt{NetMessageStat} class of \NetIbis is an example of
non-abstract subclass of \texttt{NetStat}. It can be used to record
the number of messages and the number and size (for arrays) of pieces
of data of any type used to construct these messages.


\subsection{Locks}
\label{sec:locks}

Three different synchronization classes are provided inside \NetIbis
to fit various locking schemes. The simplest one is the
\texttt{NetMutex} class which implements a \emph{semaphore of mutual
  exclusion} commonly called \emph{mutex}. A mutex is either in the
locked or in the unlocked state. If a given mutex is in the locked
state, any attempt by a given thread to lock it again will block this
thread until the mutex is unlocked by some other thread.

The \texttt{NetPriorityMutex} class is similar to the
\texttt{NetMutex} class but it introduce two levels of priority
(\texttt{low} and \texttt{high}) in the locking scheme. No pending
\texttt{low} priority locking request will be honored while one or
more \texttt{high} priority locking requests are also pending. It
should be stressed that the priority level only affects request
\emph{ordering}. In particular, no request will be honored while the
mutex is locked no matter the priority level of the pending requests.

The \texttt{NetLockArray} class allow to atomically controls the state
a set of locks. Current provided operations include locking a single
lock, locking a subset of the locks in a single step and locking the
first available lock in a subset of the locks. The locks are indexed
by an integer. The set of locks is initially empty and additional lock
indexes may be allocated at any time. A lock index may also be freed
(in which case its state is not changed). It should probably be
avoided to reallocate a lock index that as previously been deleted,
unless the programmer really know what he is doing. The indexes of the
locks are not required to be contiguous.


\subsection{Memory Allocator}
\label{sec:memory-allocator}

\NetIbis, being a communication support, obviously makes use of a lot
of memory buffers. Repeatedly allocating and freeing buffers can be
extremely expensive and either put a high pressure on the JVM's
garbage collector or may help fragmenting the heap's memory. On the
other hand, most allocated buffers are not arbitrarily sized but
instead depends on some fixed value such as the MTU of a given driver.
As a result, many buffers actually belong to series of identical size
buffers.

The \NetIbis integrated memory allocator implemented by the
\texttt{NetAllocator} class provides a means for caching buffers of
a given size and reuses previously allocated buffers each time it is
possible instead of allocating new ones. In order to use it, the
programmer only needs to construct a \texttt{NetAllocator} with the
wanted buffer size as argument. Buffers may then be allocated using
the \texttt{allocate} method and freed using the \texttt{free} method
of the newly created \texttt{NetAllocator} instance.

As with the famous \texttt{malloc}/\texttt{free} commands of the
C~language, it is important that an allocated buffer be freed once it
is of no use anymore. Failing to do so simply makes the
\texttt{NetAllocator} mechanism useless, unless the buffer is cached
somewhere else in which case it is fine to keep it allocated (but
beware of potential memory leaks in that case). 

The \texttt{DEBUG} static boolean field of the \texttt{NetAllocator}
class may be set to \texttt{true} at compile time to activate the
internal debugging support of the memory allocator. This support
allows to keep track of which method asked for a given buffer. At the
end of the session, a report is printed which lists for each remaining
not freed buffer the location in the code where it was allocated.

Similarly, the \texttt{STATISTICS} static boolean field of the
\texttt{NetAllocator} class may be set to \texttt{true} at compile
time to activate the internal buffer allocation statistics recorder,
which is of course a subclass of the \texttt{NetStat} class. It prints
the number of cached and not cached allocations as well as cached and
not cached frees\footnote{The cache size of each allocator instance is
  bounded to avoid memory explosion.}. If the debugging mode is also
activated, the statistics report also gives a comparison between the
number of allocated and the number of freed buffers.


\subsection{Buffer Factories}
\label{sec:buffer-factories}
TO-DO: Rutger

\subsection{The Bank}
\label{sec:bank}

The \NetIbis bank is a storage data structure where the process may
store objects and get one key in return for each stored object in
order to be able to access these objects later again. The bank
facility is global to a \NetIbis instance, so it is possible for one
thread to store an object and for an other thread to later access
provided that it obtained the object's key in some way. The bank can
store any \textsc{Java} \texttt{Object} and is backed by a
\texttt{HashMap} from the \textsc{Java} library. Keys are 64-bit
random values in order to limit the risk of a key misuse.



%______________________________________________________________________
\newdoublepage
\backmatter
\pagestyle{empty}
\vspace{3cm}
\begin{center}
-=-=-=-
\end{center}
\newdoublepage

\end{document}

% LocalWords:  NetIbis IPL htbp Myricom DataStream UDP udp tcp blk netibis txt
% LocalWords:  config subcontext cntxt subcntxt cntxtN subcntxtN Plugin API ipl
% LocalWords:  createIbis ibisInstance MyIbis IbisException createPortType mtu
% LocalWords:  createSendPort createReceivePort enableConnections SendPort vous
% LocalWords:  disableConnections ReceivePort newMessage WriteMessage bufferize
% LocalWords:  NetSendPort NetReceivePort NetPort ReadMessage enableUpcalls IP
% LocalWords:  disableUpcalls NetEventQueueConsumer getPortType NetEventQueue
% LocalWords:  NetEventQueueListener NetInputUpcall inputUpcall NetDriver io
% LocalWords:  getName newInput newOutput NetInput NetOutput NetBufferedOutput
% LocalWords:  initSend NetPortType setupConnection NetConnection NetSendBuffer
% LocalWords:  getServiceLink NetServiceLink getOutputSubStream sendByteBuffer
% LocalWords:  getInputSubStream NetBufferedInput NetReceiveByteBuffer doPoll
% LocalWords:  initReceive doFinish doClose doFree receiveByteBuffer rendez JDK
% LocalWords:  NetReceiveBuffer expectedLength DataOutputStream DataInputStream
% LocalWords:  PipedInputStream PipedOutputStream LocalWords HashMap NetPoller
% LocalWords:  NetSplitter closeConnection selectConnection GenSplitter NetLog
% LocalWords:  GenPoller internet NetSerializedOutput NetSerializedInput JVM
% LocalWords:  SerializationOutputStream ObjectOutputStream ArrayOutputStream
% LocalWords:  newSerializationOutputStream SerializedInputStream OutputStream
% LocalWords:  IbisSerializationOutputStream SunSerializationOutputStream JDKs
% LocalWords:  getStackTrace Throwable NetIO Disp NetStat NetMessageStat JVM's
% LocalWords:  reportVal reportMap addShutdownHook moduleName NetAllocator
% LocalWords:  malloc NetMutex mutex NetPriorityMutex NetLockArray synchronised
% LocalWords:  InterruptedException NetIbisInterruptedException
% LocalWords:  NetIbisClosedException
